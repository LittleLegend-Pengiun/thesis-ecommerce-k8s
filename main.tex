\documentclass{report}
\usepackage{vntex}
\usepackage[utf8]{inputenc}
\usepackage[unicode]{hyperref}
\usepackage{setspace}
\usepackage[a4paper,width=150mm,top=20mm,bottom=20mm,left=30mm,right=20mm]{geometry}
\usepackage{fancyhdr}
\usepackage{graphicx}
\usepackage{color}
\graphicspath{ {figures/} }
\usepackage{array}
\usepackage{float}
\usepackage{indentfirst}
\usepackage{enumerate}
\usepackage{makecell}
\usepackage[table]{xcolor}% http://ctan.org/pkg/xcolor
\usepackage{ifthen}
\usepackage{caption}
\usepackage{subcaption}
\usepackage{pdfpages}
\usepackage{minitoc}
\usepackage[toc,page,header]{appendix}
\usepackage{etoolbox}
\usepackage{longtable}
\usepackage{enumitem}


\setcounter{secnumdepth}{3}
\setcounter{tocdepth}{2}

\usepackage{titlesec}
\titleformat*{\subsubsection}{\fontsize{12pt}{12pt}\selectfont\bfseries}

\renewcommand{\chaptermark}[1]{\markboth{#1}{}}     % remove chapter number on header
\fancyhf{}                  % clear all header, footer
\fancyhead[L]{\leftmark}    % set header
\fancyfoot[C]{\thepage}     % set footer


\newcommand{\stoptocwriting}{%
  \addtocontents{toc}{\protect\setcounter{tocdepth}{-5}}}
\newcommand{\resumetocwriting}{%
  \addtocontents{toc}{\protect\setcounter{tocdepth}{\arabic{tocdepth}}}}




\appto\appendix{\addtocontents{toc}{\protect\setcounter{tocdepth}{0}}}
% reinstate the correct level for list of tables and figures
\appto\listoffigures{\addtocontents{lof}{\protect\setcounter{tocdepth}{1}}}
\appto\listoftables{\addtocontents{lot}{\protect\setcounter{tocdepth}{1}}}



% \renewcommand{\appendixtocname}{<Appendixx>}
% \addto\captionsitalian{%
%    \renewcommand{\appendixtocname}{Appendici}%
%    \renewcommand{\appendixpagename}{Appendici}%
% }



\usepackage{listings}
\usepackage{color}
\usepackage{listingsutf8}

\definecolor{dkgreen}{rgb}{0,0.6,0}
\definecolor{gray}{rgb}{0.5,0.5,0.5}
\definecolor{mauve}{rgb}{0.58,0,0.82}

% Define Javascript language
\lstdefinelanguage{JavaScript}{
  keywords={typeof, new, true, false, catch, function, return, null, catch, switch, var, if, in, while, do, else, case, break},
  keywordstyle=\color{blue}\bfseries,
  ndkeywords={class, export, boolean, throw, implements, import, this},
  ndkeywordstyle=\color{darkgray}\bfseries,
  identifierstyle=\color{black},
  sensitive=false,
  comment=[l]{//},
  morecomment=[s]{/*}{*/},
  commentstyle=\color{purple}\ttfamily,
  stringstyle=\color{red}\ttfamily,
  morestring=[b]',
  morestring=[b]"
}

\lstset{
  language=JavaScript,
  backgroundcolor=\color{lightgray},
  extendedchars=true,
  basicstyle=\footnotesize\ttfamily,
  showstringspaces=false,
  showspaces=false,
  numbers=left,
  numberstyle=\footnotesize,
  numbersep=9pt,
  tabsize=2,
  breaklines=true,
  showtabs=false,
  captionpos=b
}

% Define language terraform - kubernetes
\lstdefinelanguage{terraform}{
  keywords={terraform, variable, locals, output, module, resource, data, provider, lifecycle, provisioner, connection, count, for_each, depends_on, provider, for, if, metadata, spec, type},
  backgroundcolor=\color{white},
  keywordstyle=\color{blue}\bfseries,
  identifierstyle=\color{black},
  sensitive=false,
  comment=[l]{\#},
  commentstyle=\color{gray}\ttfamily,
  stringstyle=\color{purple}\ttfamily,
  morestring=[b]',
  morestring=[b]"
}


\lstset{basicstyle=\ttfamily,
  showstringspaces=false,
  commentstyle=\color{red},
  keywordstyle=\color{blue},
  inputencoding=utf8,
  extendedchars=true
}


\lstset{frame=tb,
  language=Java,
  aboveskip=3mm,
  belowskip=3mm,
  showstringspaces=false,
  columns=flexible,
  basicstyle={\small\ttfamily},
  numbers=none,
  numberstyle=\tiny\color{gray},
  keywordstyle=\color{blue},
  commentstyle=\color{dkgreen},
  stringstyle=\color{mauve},
  breaklines=true,
  breakatwhitespace=true,
  tabsize=3
}

% \renewcommand{\lstlistingname}{Mã nguồn}
% \renewcommand{\lstlistlistingname}{Danh sách mã nguồn}
% Docker syntax highlight
\lstdefinelanguage{docker}{
  keywords={FROM, RUN, COPY, ADD, ENTRYPOINT, CMD,  ENV, ARG, WORKDIR, EXPOSE, LABEL, USER, VOLUME, STOPSIGNAL, ONBUILD, MAINTAINER},
  backgroundcolor=\color{white},
  keywordstyle=\color{blue}\bfseries,
  identifierstyle=\color{black},
  sensitive=false,
  comment=[l]{\#},
  commentstyle=\color{purple}\ttfamily,
  stringstyle=\color{red}\ttfamily,
  morestring=[b]',
  morestring=[b]"
}

\lstdefinelanguage{docker-compose}{
  keywords={image, environment, ports, container_name, ports, volumes, links},
  keywordstyle=\color{blue}\bfseries,
  identifierstyle=\color{black},
  sensitive=false,
  comment=[l]{\#},
  commentstyle=\color{purple}\ttfamily,
  stringstyle=\color{red}\ttfamily,
  morestring=[b]',
  morestring=[b]"
}
\lstdefinelanguage{docker-compose-2}{
  % keywords={version, volumes, services},
  keywords={image, environment, ports, container_name, ports, volumes, links, working_dir, command, networks, services, restart, build, context, dockerfile, depends_on},
  keywordstyle=\color{blue}\bfseries,
  % keywords=[10]{sfu:, controller:, mongodb:, message-broker-server, event-simulator, node-api-server, node-api-gateway, redis},
  % keywordstyle=[10]\color{olive}\bfseries,
  identifierstyle=\color{black},
  sensitive=false,
  comment=[l]{\#},
  commentstyle=\color{purple}\ttfamily,
  stringstyle=\color{red}\ttfamily,
  morestring=[b]',
  morestring=[b]"
}

%%%%%%%%%%%%%%%%%%%%%%%%%%%%%%%%%%%%%%%%%%%%%%%%%%%%%%
%%%%%%%%%%% YAML syntax highlighting %%%%%%%%%%%%%%%%%

% http://tex.stackexchange.com/questions/152829/how-can-i-highlight-yaml-code-in-a-pretty-way-with-listings

% here is a macro expanding to the name of the language
% (handy if you decide to change it further down the road)
\newcommand\YAMLcolonstyle{\color{red}\mdseries}
\newcommand\YAMLkeystyle{\color{black}\bfseries}
\newcommand\YAMLvaluestyle{\color{blue}\mdseries}

\makeatletter

\newcommand\language@yaml{yaml}

\expandafter\expandafter\expandafter\lstdefinelanguage
\expandafter{\language@yaml}
{
  keywords={true,false,null,y,n},
  % assuming a key comes first
  backgroundcolor=\color{white}
  sensitive=false,
  comment=[l]{\#},
  morecomment=[s]{/*}{*/},
  commentstyle=\color{purple}\ttfamily,
  stringstyle=\YAMLvaluestyle\ttfamily,
  moredelim=[l][\color{orange}]{\&},
  moredelim=[l][\color{magenta}]{*},
  moredelim=**[il][\YAMLcolonstyle{:}\YAMLvaluestyle]{:},   % switch to value style at :
  morestring=[b]',
  morestring=[b]",
  literate =    {---}{{\ProcessThreeDashes}}3
                {>}{{\textcolor{red}\textgreater}}1     
                {|}{{\textcolor{red}\textbar}}1 
                {\ -\ }{{\mdseries\ -\ }}3,
}

% switch to key style at EOL
\lst@AddToHook{EveryLine}{\ifx\lst@language\language@yaml\YAMLkeystyle\fi}
\makeatother

\newcommand\ProcessThreeDashes{\llap{\color{cyan}\mdseries-{-}-}}

%%%%%%%%%%% YAML syntax highlighting %%%%%%%%%%%%%%%%%
%%%%%%%%%%%%%%%%%%%%%%%%%%%%%%%%%%%%%%%%%%%%%%%%%%%%%%

\onehalfspacing
\begin{document}

\begingroup
\fontsize{12pt}{12pt}\selectfont

%%%%%%%%%%%%%%%%%%%%% LVTN %%%%%%%%%%%%%%%%%%%%%
% \newpage
% \thispagestyle{empty}
% % Phiếu nhiệm vụ
% \includepdf[pages={1}]{pdf/phieunhiemvu.pdf}
% \newpage
% \thispagestyle{empty}
% % Phiếu chấm hướng dẫn
% \includepdf[pages=-]{pdf/phieuchamhuongdan.pdf}
% \newpage
% \thispagestyle{empty}
% % Phiếu phản biện
% \includepdf[pages=-]{pdf/phieuchamphanbien.pdf}

% \newpage
% % \includepdf[pages=-]{pdf/cover.pdf}
% \includepdf[pages=-]{pdf/cover_GD2 - final - Copy.pdf}


% \newpage
% \thispagestyle{empty}
% % Phiếu nhiệm vụ
% \includepdf[pages={1}]{pdf/phieunhiemvu.pdf}
% \newpage
% \thispagestyle{empty}
% % Phiếu chấm hướng dẫn
% \includepdf[pages=-]{pdf/phieuchamhuongdan.pdf}
% \newpage
% \thispagestyle{empty}
% % Phiếu phản biện
% \includepdf[pages=-]{pdf/phieuchamphanbien.pdf}

% \newpage
% \thispagestyle{empty}
% \input{sections/signatures}

\pagenumbering{roman}
\newpage
\begin{titlepage}

    \begingroup
        \fontsize{15pt}{12pt}\selectfont
        \begin{center}
            \textbf{
                ĐẠI HỌC QUỐC GIA THÀNH PHỐ HỒ CHÍ MINH \\
                TRƯỜNG ĐẠI HỌC BÁCH KHOA \\
                KHOA KHOA HỌC VÀ KỸ THUẬT MÁY TÍNH
            }
        \end{center}       
    \endgroup
    
    \vspace{1cm}
    
    \begin{figure}[h!]
        \begin{center}
            \includegraphics[width=4cm]{images/hcmut.png}
        \end{center}
    \end{figure}
    
    \vspace{0.5cm}

    \begingroup
        \fontsize{15pt}{12pt}\selectfont
        \begin{center}
            \textbf{BÁO CÁO} \\
            \textbf{ĐỒ ÁN CHUYÊN NGÀNH}
        \end{center}       
    \endgroup

    \vspace{0.5cm}


    \begingroup
        \fontsize{18pt}{12pt}\selectfont
        \begin{center}
            \textbf{PHÁT TRIỂN HỆ THỐNG THƯƠNG MẠI ĐIỆN TỬ CÓ TÍNH SẴN SÀNG VÀ MỞ RỘNG CAO CHO SẢN PHẨM CÔNG NGHỆ}
        \end{center}       
    \endgroup

    \vspace{0.5cm}

    \begingroup
        \fontsize{15pt}{12pt}\selectfont
        \begin{center}
            Ngành: Khoa học Máy tính
        \end{center}       
    \endgroup
    
    \vspace{1.5cm}

    \begingroup
        \fontsize{14pt}{12pt}\selectfont
        \begin{center}
            \begin{tabular}{rll}
                \color{black} \textbf{Hội đồng:} & \color{black} Đồ án chuyên ngành &  \\
                \color{black} \textbf{Giảng viên hướng dẫn:} & \color{black} Phan Trọng Nhân &  \\

                \\

                \multicolumn{3}{c}{\noindent\rule{4cm}{0.5pt} \textbf{oOo} \noindent\rule{4cm}{0.5pt}} \\ \\
                \color{black}\textbf{Sinh viên thực hiện 1:} & \color{black}Lê Hoàng Anh & \color{black}1910752 \\
                \color{black}\textbf{Sinh viên thực hiện 2:} & \color{black}Hoàng Văn Hiếu & \color{black}1913328 \\
                \color{black}\textbf{Sinh viên thực hiện 3:} & \color{black}Thới Duy Phát & \color{black}2120049 \\
            \end{tabular}
        \end{center}
    \endgroup
    
    \vspace{1.5cm}

    \begingroup
        \fontsize{12pt}{12pt}\selectfont
        \begin{center}
            {TP.Hồ Chí Minh, Tháng 12/2023}
        \end{center}
    \endgroup
    
\end{titlepage}

\newpage
\tableofcontents

\newpage
\thispagestyle{empty}
\listoftables
\listoffigures
% \lstlistoflistings

\newpage

\pagestyle{fancy}



\pagenumbering{arabic}
\chapter{Giới thiệu}
% \section{Ví dụ}

% \noindent Đây là một ví dụ, sau này sẽ được update nội dung sau.

% If this chapter/section has a star, it won't be in the table of content.
% \section*{Đây là section k dc thêm vào mục lục}
\section{Giới thiệu đề tài}
\noindent Trong những năm gần đây, với sự bùng nổ và đại trà của các thiết bị điện tử, cùng với ảnh hưởng của đại dịch, người dùng ngày càng có xu hướng mua sắm online thay vì đi ra cửa hàng trực tiếp. Do đó, việc một doanh nghiệp sở hữu kênh bán hàng trên nền tảng số là vô cùng cần thiết. Mục tiêu của đề tài là xây dựng một hệ thống thương mại điện tử có tính sẵn sàng, tính mở rộng, tính ổn định cao, có thể được đưa lên nhiều nền tảng khác nhau một cách dễ dàng, dễ bảo trì, nâng cấp trong tương lai. \\

\noindent Hệ thống bao gồm những chức năng chính cho một trang web thương mại điện tử bán sản phẩm công nghệ và một số tính năng, đặc điểm nổi bật như tính sẵn sàng cao, tính mở rộng cao, đáp ứng được lưu lượng truy cập biến động của người dùng.

\section{Mục tiêu và phạm vi của đề tài}
% \noindent Mục

\section{Cấu trúc đồ án}
\noindent Nội dung của đồ án được trình bày 

\chapter{Cơ sở lý thuyết}
\section{Các cơ sở lý thuyết và công nghệ sử dụng}
\subsection{Reactjs}
\subsubsection{Khái niệm}
\indent ReactJS là một thư viện JavaScript phía người dùng (frontend) được sử dụng để xây dựng giao diện người dùng tương tác.
\subsubsection{Ưu điểm của Reactjs}
\begin{itemize}
    \item \textbf{Tận dụng lại các thành phần có sẵn}

    \indent ReactJS hỗ trợ tích cực trong khởi tạo một website bởi lập trình viên sẽ không cần phải code nhiều như khi tạo trang web mà chỉ sử dụng JavaScript. Đồng thời, nó cung cấp một loạt các thành phần sẵn có mà bạn có thể sử dụng trong nhiều tình huống khác nhau.
    \item \textbf{Tích hợp được cho cả ứng dụng di động Mobile application}

    \indent Hầu hết chúng ta đã biết rằng ReactJS được sử dụng để phát triển các ứng dụng web, tuy nhiên, nó không chỉ giới hạn trong lĩnh vực đó. Nếu chúng ta muốn phát triển các ứng dụng di động, chúng ta có thể sử dụng React Native. Đây là một framework do Facebook phát triển, cho phép chúng ta dễ dàng "chia sẻ" các thành phần và tái sử dụng logic nghiệp vụ trong các ứng dụng của chúng ta.
    \item \textbf{Tối ưu để tăng cường khả năng tìm kiếm SEO}

    \indent Tối ưu hóa công cụ tìm kiếm (SEO) là một yếu tố quan trọng để đảm bảo trang web của chúng ta xuất hiện cao hơn trong kết quả tìm kiếm của Google. ReactJS là một thư viện JavaScript cơ bản. Công cụ tìm kiếm của Google có khả năng thu thập thông tin và lập chỉ mục mã JavaScript, tuy nhiên, nó cũng yêu cầu sự hỗ trợ từ các thư viện khác để làm điều này.
    \item \textbf{Dễ dàng sửa lỗi và gỡ rối Debug}

    \indent Facebook đã phát hành một tiện ích mở rộng Chrome để hỗ trợ việc gỡ lỗi trong quá trình phát triển ứng dụng. Điều này giúp tăng tốc quá trình phát hành sản phẩm cũng như quá trình viết mã của chúng ta.
\end{itemize}
\subsubsection{Nhược điểm của Reactjs}
\begin{itemize}
    \item Reactjs không phải là framework, cho nên chúng ta phải tự xây dựng dự án bằng thủ công.
    \item Tích hợp Reactjs vào các framework MVC truyền thống yêu cầu cần phải cấu hình lại.
    \item Poor Document: Đó là một nhược điểm khá phổ biến đối với các công nghệ cập nhật liên tục. Các công nghệ cập nhật và tăng tốc nhanh đến mức không có thời gian để tạo tài liệu phù hợp.
\end{itemize}
\subsection{Java Spring Boot}

\subsubsection{Khái niệm:} Spring Boot là một framework phát triển ứng dụng Java, dựa trên nền tảng Spring Framework. Nó được thiết kế để giảm bớt công việc cấu hình và cung cấp một cách tiếp cận linh hoạt và nhanh chóng để xây dựng các ứng dụng Java.

\subsubsection{Ưu điểm:}
\begin{itemize}
    \item \textbf{Tự động cấu hình:} Spring Boot tự động cấu hình dựa trên các quy tắc mặc định và cung cấp một số cấu hình tùy chỉnh đơn giản.
    \item \textbf{Tiết kiệm thời gian:} Với Spring Boot, bạn không cần lo lắng về việc cấu hình chi tiết và tập trung vào việc phát triển chức năng của ứng dụng.
    \item \textbf{Tích hợp dễ dàng:} Spring Boot tích hợp tốt với các công nghệ và thư viện phổ biến khác, cho phép bạn dễ dàng tích hợp các thành phần khác như cơ sở dữ liệu, bảo mật, gửi email, vv.
    \item \textbf{Gói hóa ứng dụng:} Spring Boot cho phép bạn gói hóa ứng dụng thành file JAR hoặc WAR, giúp dễ dàng triển khai và chạy trên các môi trường khác nhau.
\end{itemize}

\subsubsection{Nhược điểm:}
\begin{itemize}
    \item \textbf{Phụ thuộc:} Spring Boot có thể tạo ra các ứng dụng có phụ thuộc tăng cao vào framework, điều này có thể làm cho ứng dụng phải dựa vào các phiên bản và cấu hình cụ thể của Spring Boot.
    \item \textbf{Tính linh hoạt:} Mặc dù Spring Boot giảm bớt công việc cấu hình, nhưng đôi khi nó có thể hạn chế tính linh hoạt so với việc cấu hình thủ công bằng XML hoặc Java.
    \item \textbf{Độ phức tạp:} Một số tính năng và cấu hình cao cấp của Spring Boot có thể trở nên phức tạp và khó hiểu đối với những người mới sử dụng.
\end{itemize}
\indent Tuy nhiên, ưu điểm của Spring Boot thường vượt trội hơn so với nhược điểm, vì nó giúp đơn giản hóa việc phát triển và triển khai ứng dụng Java.

\subsection{Nodejs}
\subsubsection{Khái niệm}
\indent Node.js là một môi trường chạy mã JavaScript phía máy chủ, dựa trên JavaScript Engine V8 của Google. Nó cho phép viết mã JavaScript để xây dựng ứng dụng máy chủ một cách hiệu quả.
\subsubsection{Ưu điểm của Node.js:}
\begin{itemize}
    \item Hiệu suất cao: Với JavaScript Engine V8 nhanh chóng, Node.js cho phép xử lý các yêu cầu đồng thời một cách hiệu quả và đạt được hiệu suất cao.
    \item Đơn luồng và không đồng bộ: Node.js sử dụng mô hình xử lý không đồng bộ (non-blocking) I/O, giúp xử lý nhiều yêu cầu cùng một lúc mà không tốn thêm tài nguyên.
    \item Quản lý gói dễ dàng: Node.js có npm (Node Package Manager), cung cấp một kho lưu trữ gói phong phú và dễ quản lý.
    \item Phát triển đồng nhất: Với Node.js, phát triển ứng dụng web và ứng dụng di động có thể được thực hiện bằng cùng một ngôn ngữ và công cụ, tạo sự đồng nhất.
\end{itemize}
\subsubsection{Nhược điểm của Node.js:}
\begin{itemize}
    \item Chưa phù hợp cho các tác vụ nặng: Do Node.js sử dụng mô hình đơn luồng, nó không phù hợp cho các tác vụ tính toán nặng hoặc xử lý dữ liệu lớn.
    \item Đòi hỏi khéo léo trong việc quản lý bộ nhớ: Node.js không tự động quản lý bộ nhớ, điều này yêu cầu phải làm việc thủ công để tránh rò rỉ bộ nhớ.
\end{itemize}
\subsection{Docker}
\subsubsection{Khái niệm:}
\indent Docker là một nền tảng mã nguồn mở giúp đóng gói các ứng dụng và các phụ thuộc của chúng vào những đơn vị gọi là containers. Containers cho phép triển khai một ứng dụng một cách đáng tin cậy và di động với sự độc lập về môi trường.
\subsubsection{Ưu điểm:}
\begin{itemize}
    \item Đóng gói và triển khai dễ dàng: Docker giúp đóng gói các ứng dụng và phụ thuộc vào một container có thể di chuyển được và triển khai một cách dễ dàng trên nhiều môi trường khác nhau.
    \item Tính nhất quán giữa môi trường phát triển và triển khai: Docker đảm bảo môi trường chạy ứng dụng trên máy chủ phát triển giống với môi trường chạy trên môi trường triển khai.
    \item Hiệu suất cao: Containers Docker nhẹ và nhanh chóng, giúp tối ưu hóa tài nguyên và cung cấp hiệu suất cao cho các ứng dụng.
\end{itemize}
\subsubsection{Nhược điểm:}
\begin{itemize}
    \item Tăng phức tạp: Đôi khi quản lý các container Docker và xử lý các phụ thuộc có thể trở nên phức tạp, đặc biệt là trong các môi trường lớn và phức tạp.
    \item Hiệu suất ảnh hưởng: Mặc dù Docker giúp tối ưu hóa hiệu suất, nhưng việc sử dụng container cũng có thể ảnh hưởng đến hiệu suất so với việc chạy ứng dụng trực tiếp trên máy chủ vật lý.
\end{itemize}

\subsection{Terraform}
\subsubsection{Khái niệm:}
\indent Terraform là một công cụ mã hóa cấu hình (Infrastructure as Code) được sử dụng để tự động hóa việc triển khai và quản lý cơ sở hạ tầng đám mây và hạ tầng điện toán.
\subsubsection{Ưu điểm:}
\begin{itemize}
    \item Tự động hóa hạ tầng: 
    Terraform cho phép viết mã để mô tả và triển khai cơ sở hạ tầng một cách dễ dàng và nhất quán.
    \item Đa nền tảng: 
    Terraform hỗ trợ nhiều nhà cung cấp đám mây và nền tảng hạ tầng khác nhau như AWS, Azure, GCP, v.v., giúp quản lý và triển khai đồng nhất trên nhiều môi trường.
    \item Kiểm soát phiên bản: 
    Terraform quản lý các tài nguyên hạ tầng như mã nguồn, cho phép theo dõi và quản lý phiên bản tài nguyên trong quá trình phát triển và triển khai.
\end{itemize}
\subsubsection{Nhược điểm:}
\begin{itemize}
    \item Học và quản lý đòi hỏi thời gian: Terraform có độ dốc học đôi khi khá lớn, và việc quản lý mã cấu hình có thể đòi hỏi thời gian và kỹ năng.
    \item Giới hạn của các nhà cung cấp đám mây: Các nhà cung cấp đám mây có thể không hỗ trợ tất cả các tính năng của Terraform hoặc có giới hạn trong việc quản lý hạ tầng.
\end{itemize}
\subsection{Kubernetes}
\subsubsection{Khái niệm:}
\indent Kubernetes (thường được gọi là k8s) là một nền tảng mã nguồn mở để quản lý việc triển khai, tự động hóa và mở rộng ứng dụng container.

\indent Các ứng dụng có sử dụng Kubernetes: Google, Netflix, Airbnb, Spotify, Grab, Zalando, Adidas...và nhiều hơn nữa. Kubernetes đã trở thành một công nghệ phổ biến và mạnh mẽ trong việc quản lý và triển khai ứng dụng quy mô lớn.
\subsubsection{Kiến trúc}
 \begin{figure}[H]
    \begin{center}
    \includegraphics[scale = 0.24]{images/chap-2-images/kubernetes_architecture.jpg}
    \vspace*{7mm}
    \caption{Kubernetes Architecture}
    \end{center}
    \label{}
\end{figure}
\indent Kiến trúc của Kubernetes bao gồm:
\begin{itemize}
    \item Master Node: Quản lý, điều phối và giám sát toàn bộ hệ thống Kubernetes.
    \item Worker Node: Chứa các container và chịu trách nhiệm thực hiện các tác vụ đồng bộ từ Master Node.
    \item Pod: Nhóm các container chạy cùng nhau trên cùng một Worker Node, chia sẻ tài nguyên và mạng.
    \item Service: Một tập hợp các Pod có thể truy cập thành một đầu nối duy nhất từ bên ngoài.
    \item Volume: Cung cấp quản lý lưu trữ cho các container trong Pod.
\end{itemize}
\subsubsection{Cách thiết lập Kubernetes cho ứng dụng}
\begin{itemize}
    \item Định nghĩa và triển khai mô tả ứng dụng: Sử dụng các tệp cấu hình (ví dụ: YAML) để định nghĩa ứng dụng, bao gồm Pod, Service và các tài nguyên khác.
    \item Triển khai và quản lý: Gửi yêu cầu triển khai tới Master Node, sau đó Kubernetes sẽ triển khai các Pod và Service, quản lý vòng đời và giám sát.
    \item Tự động mở rộng và cân bằng tải: Kubernetes tự động mở rộng hoặc thu hẹp quy mô của các Pod để đáp ứng tải công việc và cân bằng tải giữa các Worker Node.
\end{itemize}
\subsubsection{Ưu điểm:}
\begin{itemize}
    \item Tự động hóa và quản lý quy mô: Kubernetes cho phép tự động mở rộng và thu hẹp quy mô các thành phần ứng dụng, như các microservice và cụm máy chủ. Điều này giúp tối ưu hóa hiệu suất và đáp ứng đối với lưu lượng truy cập thay đổi.

    \item Đảm bảo sẵn lòng và tin cậy: Kubernetes có khả năng phục hồi lỗi tự động và chuyển đổi dịch vụ giữa các phiên bản trên các nút khác nhau. Điều này giúp đảm bảo rằng ứng dụng luôn sẵn sàng và hoạt động một cách tin cậy.

    \item Quản lý tài nguyên hiệu quả: Kubernetes cung cấp các công cụ quản lý tài nguyên để phân bổ và giám sát tài nguyên phù hợp, bao gồm bộ nhớ, CPU, lưu trữ và mạng. Điều này giúp tối ưu hóa sử dụng tài nguyên và tăng hiệu suất hệ thống.

\end{itemize}
\subsubsection{Nhược điểm:}
\begin{itemize}
    \item Đòi hỏi kiến thức phức tạp: Kubernetes có độ dốc học và quản lý phức tạp, đòi hỏi kiến thức về hạ tầng và kỹ năng quản lý container.
    \item Tài nguyên tốn kém: Kubernetes yêu cầu sự sẵn có của một cụm máy chủ và tài nguyên đáng kể để triển khai và vận hành.
\end{itemize}
\subsection{Redis Cache}
\subsubsection{Khái niệm:}
\indent Redis Cache là một cơ sở dữ liệu key-value (khóa-giá trị) phân tán, được sử dụng để lưu trữ dữ liệu tạm thời trong bộ nhớ.
\subsubsection{Ưu điểm:}
\begin{itemize}
    \item Tốc độ và hiệu suất cao: Redis Cache lưu trữ dữ liệu trong bộ nhớ và cho phép truy cập cực kỳ nhanh chóng, đáp ứng yêu cầu với hiệu suất cao.
    \item Đa dạng tính năng: Redis cung cấp nhiều tính năng như caching, xử lý hàng đợi, pub/sub messaging và phân tích dữ liệu, giúp tối ưu hóa các tác vụ dựa trên dữ liệu.
\end{itemize}
\subsubsection{Nhược điểm:}
\begin{itemize}
    \item Giới hạn bộ nhớ: Redis Cache yêu cầu bộ nhớ đủ lớn để lưu trữ dữ liệu. Nếu dữ liệu vượt quá dung lượng bộ nhớ, có thể gặp sự cố và ảnh hưởng đến hiệu suất.
    \item Khả năng mất dữ liệu: Redis Cache mặc định không cung cấp cơ chế đồng bộ hoá dữ liệu, điều này đồng nghĩa rằng có thể mất dữ liệu khi xảy ra sự cố.
\end{itemize}
\subsection{PostgreSQL}
\subsubsection{Khái niệm:}
\indent PostgreSQL (viết tắt là Postgres) là một hệ quản trị cơ sở dữ liệu quan hệ mã nguồn mở, được đánh giá là ổn định, mạnh mẽ và có tính mở rộng.
\subsubsection{Ưu điểm:}
\begin{itemize}
    \item Độ tin cậy cao: PostgreSQL được thiết kế để đảm bảo tính toàn vẹn và độ tin cậy của dữ liệu, bao gồm các tính năng như ACID và khả năng khôi phục dữ liệu.
    \item Tính mở rộng và phân vùng: PostgreSQL hỗ trợ phân vùng dữ liệu và khả năng mở rộng sẵn sàng, cho phép mở rộng cơ sở dữ liệu để xử lý lượng dữ liệu lớn và tải cao.
    \item Đa dạng tính năng: PostgreSQL cung cấp nhiều tính năng tiên tiến bao gồm trình tự, trigger, tìm kiếm văn bản và hình ảnh, và hỗ trợ các loại dữ liệu phong phú.
\end{itemize}
\subsubsection{Nhược điểm:}
\begin{itemize}
    \item Có thể cảm thấy phức tạp đối với các dự án nhỏ.
\end{itemize}
\subsection{EKS}
\subsubsection{Khái niệm:}
\indent EKS (Elastic Kubernetes Service) là một dịch vụ quản lý Kubernetes do Amazon Web Services (AWS) cung cấp.
\subsubsection{Ưu điểm:}
\begin{itemize}
    \item Dễ dàng triển khai, quản lý và mở rộng các ứng dụng chạy trên Kubernetes.
    \item Tích hợp tốt với dịch vụ AWS khác.
    \item Hỗ trợ cho môi trường đám mây tiêu chuẩn và quy mô lớn.
\end{itemize}
\subsubsection{Nhược điểm:}
\begin{itemize}
    \item Phí sử dụng có thể cao, đặc biệt trong trường hợp triển khai lớn.
    \item Đòi hỏi kiến thức về quản lý và triển khai hệ thống phức tạp hơn so với các giải pháp khác.
\end{itemize}
\subsection{AKS}
\subsubsection{Khái niệm:}
\indent AKS (Azure Kubernetes Service) là một dịch vụ quản lý Kubernetes.

\subsubsection{Ưu điểm:}
\begin{itemize}
    \item Dễ dàng triển khai, quản lý và mở rộng các ứng dụng chạy trên Kubernetes.
    \item Tích hợp tốt với dịch vụ Azure và công cụ phát triển của Microsoft.
    \item Cung cấp tính năng bảo mật và giám sát mạnh mẽ.
\end{itemize}
\subsubsection{Nhược điểm:}
\begin{itemize}
    \item Phí sử dụng có thể cao, đặc biệt trong trường hợp triển khai lớn.
    \item Yêu cầu sử dụng môi trường và công cụ phát triển Azure.
\end{itemize}
\subsection{RabbitMQ}
\subsubsection{Khái niệm:}
\indent RabbitMQ là một hệ thống message broker mã nguồn mở, dựa trên giao thức AMQP (Advanced Message Queuing Protocol).
\subsubsection{Kiến trúc của RabbitMQ}
\begin{itemize}
    \item Producer:

Là thành phần tạo ra và gửi các thông điệp (message).
Message được gửi đến một exchange.
    \item Exchange:

Nhận thông điệp từ nhà sản xuất và định tuyến chúng đến hàng đợi (queue) thích hợp.
Có các loại định tuyến khác nhau như direct, topic, fanout, và headers, cho phép định tuyến dựa trên các tiêu chí khác nhau.
    \item Queue (Hàng đợi):

Là nơi lưu trữ các thông điệp đến từ sàn giao dịch cho đến khi chúng được xử lý bởi một tiêu thụ (consumer).
Các hàng đợi có thể được chia sẻ giữa nhiều tiêu thụ hoặc có thể chỉ được sử dụng bởi một tiêu thụ cụ thể.
    \item Binding (Ràng buộc):

Liên kết giữa một exchange và một queue, xác định cách thông điệp nên được định tuyến từ exchange đến queue.
Ràng buộc này được thiết lập thông qua các quy tắc định tuyến (routing key).
    \item Consumer:

Là thành phần đọc và xử lý các thông điệp từ hàng đợi.
Có thể có nhiều tiêu thụ cùng một lúc đọc từ cùng một hàng đợi.
    \item Virtual Host:

Là một không gian làm việc ảo trong RabbitMQ, giúp tách biệt và cô lập các ứng dụng và người dùng khác nhau.
Mỗi Virtual Host có thể có các exchange, queue, và quyền riêng biệt.
    \item Broker:

Là nền tảng hoạt động của RabbitMQ.
Nhận thông điệp từ nhà sản xuất, định tuyến chúng đến hàng đợi thông qua sàn giao dịch và chuyển giao chúng đến các tiêu thụ.
    \item Connection:

Là một kết nối mạng giữa ứng dụng và RabbitMQ Broker.
Mỗi ứng dụng có thể có nhiều kết nối.
\end{itemize}

\subsubsection{Ưu điểm:}
\begin{itemize}
    \item Hỗ trợ đa ngôn ngữ và dễ dàng tích hợp với các ứng dụng phổ biến.
    \item Cung cấp tính năng đám mây phân tán, đảm bảo bất đồng bộ và xử lý hàng đợi.
    \item Tích hợp tốt với các công nghệ và framework khác như Spring, .NET, Node.js, etc.
\end{itemize}
\subsubsection{Nhược điểm:}
\begin{itemize}
    \item Cấu hình phức tạp và yêu cầu kiến thức về hệ thống phân tán.
    \item Hiệu suất có thể bị ảnh hưởng đối với tải công việc rất cao.
\end{itemize}
\chapter{Phân tích yêu cầu}
\section{Phân tích nghiệp vụ}
\section{Phân tích bài toán và đề xuất giải pháp}
\subsection{Định nghĩa các yêu cầu mục tiêu của đồ án}
\noindent Đề tài xây dựng nhằm phân tích và giải quyết bài toán xoay quanh 2 từ khóa \textbf{scalability} và \textbf{availability}. Do đó, ta cần đi vào tìm hiểu định nghĩa của 2 từ khóa này:
% \\[0.5cm]

\subsubsection{Availability - Tính sẵn sàng}
\noindent Theo định nghĩa của Microsoft\footnote{Website: https://learn.microsoft.com/en-us/training/modules/describe-benefits-use-cloud-services/2-high-availability-scalability-cloud}, “Khi bạn deploy một ứng dụng, dịch vụ, hay bất kỳ tài nguyên IT nào, việc những tài nguyên đó sẵn sàng khi bạn cần là điều quan trọng. High availability tập trung vào việc đảm bảo tối đa tính sẵn sàng của hệ thống, bất kể sự gián đoạn hay sự kiện nào có thể xảy ra.” \\[0.5cm]
\noindent Khi xây dựng giải pháp của mình, ta sẽ cần tính đến các đảm bảo về tính khả dụng của dịch vụ. Azure là môi trường đám mây có tính sẵn sàng cao với sự đảm bảo về thời gian hoạt động tùy thuộc vào dịch vụ. Những đảm bảo này là một phần của thỏa thuận cấp độ dịch vụ (SLA).

\subsubsection{Scalability - Tính mở rộng}
\noindent Theo định nghĩa của Microsoft\footnote{Như trên}, “Khả năng mở rộng (scalability) đề cập đến khả năng điều chỉnh các tài nguyên để đáp ứng nhu cầu. Nếu bạn đột nhiên gặp phải lưu lượng truy cập cao và hệ thống của bạn bị quá tải thì khả năng mở rộng quy mô có nghĩa là bạn có thể bổ sung thêm tài nguyên để xử lý tốt hơn lượng tải đang gia tăng.” \\[0.5cm]
\noindent Lợi ích khác của tính mở rộng là bạn không phải trả quá nhiều tiền cho các dịch vụ. Vì đám mây là mô hình dựa trên mức tiêu dùng nên bạn chỉ trả tiền cho những gì bạn sử dụng. Nếu nhu cầu giảm, bạn có thể giảm tài nguyên và từ đó giảm chi phí. \\[0.5cm]
\noindent Scalability thường có hai loại: dọc và ngang. Mỏ rộng theo chiều dọc tập trung vào việc tăng hoặc giảm khả năng của tài nguyên. Mở rộng theo chiều ngang là thêm hoặc bớt số lượng tài nguyên.
\subsection{Phân tích các kiểu kiến trúc hệ thống}
\subsubsection{Kiến trúc monolith}
\noindent \textbf{Định nghĩa} \\[0.3cm]
\noindent Kiến trúc monolith là kiến trúc trong đó tất cả các thành phần của một ứng dụng được đặt trong một đơn vị duy nhất. Đơn vị này thường bị hạn chế trong một phiên bản thời gian chạy duy nhất của ứng dụng. Các ứng dụng truyền thống thường bao gồm giao diện web, lớp dịch vụ và lớp dữ liệu. Trong kiến trúc monolith, các lớp này được kết hợp trên một phiên bản của ứng dụng.\footnote{Website: https://learn.microsoft.com/en-us/training/modules/microservices-architecture/}\\[0.5cm]
\textbf{Lý do sử dụng}\\[0.5cm]
Kiến trúc monolith thường là giải pháp phù hợp cho các ứng dụng nhỏ, nhưng chúng có thể trở nên khó sử dụng khi ứng dụng phát triển. Ban đầu, một ứng dụng nhỏ có thể nhanh chóng trở thành một hệ thống phức tạp, khó mở rộng quy mô, khó triển khai và khó đổi mới.\\[0.5cm]
\textbf{Thách thức}\\[0.5cm]
Tất cả các dịch vụ được chứa trong một đơn vị duy nhất. Sự sắp xếp này mang lại những thách thức khi hoạt động kinh doanh của họ và tải hệ thống tiếp theo phát triển. Một số thách thức này là:
\begin{itemize}
    \item Khó mở rộng quy mô dịch vụ một cách độc lập.
    \item Phức tạp để phát triển và quản lý việc triển khai khi cơ sở mã phát triển, điều này làm chậm quá trình phát hành và triển khai tính năng mới.
    \item Kiến trúc được gắn với một ngăn xếp công nghệ duy nhất, điều này hạn chế sự đổi mới trong các nền tảng và SDK mới.
    \item Cập nhật lược đồ dữ liệu có thể ngày càng khó khăn.
\end{itemize}
Những thách thức này có thể được giải quyết bằng cách xem xét các kiến trúc thay thế, chẳng hạn như kiến trúc microservices.
\subsubsection{Kiến trúc microservices}
\noindent \textbf{Định nghĩa}\\[0.5cm]
Kiến trúc microservice bao gồm các dịch vụ nhỏ, độc lập và được liên kết lỏng lẻo. Mỗi dịch vụ có thể được triển khai và mở rộng quy mô một cách độc lập.\\[0.5cm]
Một microservice đủ nhỏ để một nhóm nhỏ các nhà phát triển có thể viết và duy trì nó. Vì các dịch vụ có thể được triển khai độc lập nên một nhóm có thể cập nhật dịch vụ hiện có mà không cần xây dựng lại và triển khai lại toàn bộ ứng dụng.\\[0.5cm]
Mỗi dịch vụ thường chịu trách nhiệm về dữ liệu riêng của mình. Cấu trúc dữ liệu của nó được tách biệt nên việc nâng cấp hoặc thay đổi lược đồ không phụ thuộc vào các dịch vụ khác. Các yêu cầu về dữ liệu thường được xử lý thông qua API và cung cấp mô hình truy cập nhất quán và được xác định rõ ràng. Chi tiết triển khai nội bộ được ẩn khỏi người tiêu dùng dịch vụ.\\[0.5cm]
Vì mỗi dịch vụ đều độc lập nên chúng có thể sử dụng các nhóm công nghệ, khung và SDK khác nhau. Người ta thường thấy các dịch vụ dựa vào các lệnh gọi REST để liên lạc giữa các dịch vụ bằng cách sử dụng các API được xác định rõ ràng thay vì các lệnh gọi thủ tục từ xa (RPC) hoặc các phương thức liên lạc tùy chỉnh khác.\\[0.5cm]
Kiến trúc vi dịch vụ không phụ thuộc vào công nghệ, nhưng bạn thường thấy các bộ chứa hoặc công nghệ serverless được sử dụng để triển khai chúng. Triển khai liên tục và tích hợp liên tục (CI/CD) thường được sử dụng để tăng tốc độ và chất lượng của các hoạt động phát triển.\\[0.5cm]
\textbf{Lý do sử dụng}\\[0.5cm]
Có một số lợi ích chính đối với kiến trúc microservices:
\begin{itemize}
    \item Nhanh nhẹn
    \item Mã nhỏ, nhóm nhỏ
    \item Sự kết hợp của công nghệ
    \item khả năng phục hồi
    \item Khả năng mở rộng (Scalability)
    \item Cách ly dữ liệu
\end{itemize}
\textbf{Những thách thức}\\[0.5cm]
Có rất nhiều lợi ích đối với kiến trúc microservices, nhưng đó không phải là tất cả. Kiến trúc microservice có những thách thức riêng:
\begin{itemize}
    \item Độ phức tạp
    \item Phát triển và thử nghiệm
    \item Thiếu quản trị
    \item Tắc nghẽn mạng và độ trễ
    \item Toàn vẹn dữ liệu
    \item Sự quản lý
    \item Phiên bản
    \item Bộ kỹ năng    
\end{itemize}
\textbf{Khi nào nên chọn kiến trúc microservices?}\\[0.5cm]
Dựa vào những thông tin trên, kiến trúc microservices sẽ phù hợp trong những tình huống sau
\begin{itemize}
    \item Các ứng dụng lớn đòi hỏi tốc độ phát hành cao.
    \item Các ứng dụng phức tạp cần có khả năng mở rộng cao.
    \item Ứng dụng có miền phong phú hoặc nhiều miền phụ.
    \item Một tổ chức bao gồm các nhóm phát triển nhỏ.    
\end{itemize}
\subsection{Một số bài viết và ví dụ}

\subsection{Đề xuất giải pháp}
\chapter{Thiết kế hệ thống}
\section{Thiết kế giao diện bằng ứng dụng Figma}
\subsection{Thiết kế giao diện trang hiển thị sản phẩm}
\begin{figure}[H]
    \begin{center}
    \includegraphics[scale=0.8]{images/hieu/chap-4/display-product-page.png}
    \vspace*{5mm}
    \caption{Thiêt kế giao diện trang hiển thị sản phẩm}
    \end{center}
\end{figure}
\begin{itemize}
    \item \textbf{Phần Header}
    \newline
    Phần đầu trang web gồm các thành phần sau:
    \begin{itemize}
        \item Logo của trang web
        \item Tên trang web
        \item Thanh điều hướng chứa các nút điều hướng đến các trang khác
        \item Ô tìm kiếm sản phẩm
        \item Nút đăng nhập
        \item Giỏ hàng
    \end{itemize}
    \begin{figure}[H]
        \begin{center}
        \includegraphics[scale=0.5]{images/hieu/chap-4/header.png}
        \vspace*{5mm}
        \caption{Phần đầu trang - Header}
        \end{center}
    \end{figure}

    \item \textbf{Phần Category}
    \newline
    Phần category là một thanh dropdown chứa các danh mục sản phẩm.
    \begin{figure}[H]
        \begin{center}
        \includegraphics[scale=1]{images/hieu/chap-4/category.png}
        \vspace*{5mm}
        \caption{Phần danh mục sản phẩm - Category}
        \end{center}
    \end{figure}
    \item \textbf{Phần Discount}
    \newline
    Phần discount là một slide chứa các giảm giá của các sản phẩm. Slide sẽ tự chuyển động sau một khoảng thời gian nhất định hoặc có thể chuyển động bằng cách nhấn vào các nút điều hướng.
    \begin{figure}[H]
        \begin{center}
        \includegraphics[scale=0.7]{images/hieu/chap-4/discount.png}
        \vspace*{5mm}
        \caption{Phần giảm giá - Discount}
        \end{center}
    \end{figure}
    \item \textbf{Phần Product}
    \newline
    Phần product hiển thị các sản phẩm theo danh mục. Mỗi sản phẩm gồm có:
    \begin{itemize}
        \item Hình ảnh sản phẩm
        \item Tên sản phẩm
        \item Giá sản phẩm
        \item Nút thêm vào giỏ hàng
        \item Nút xem chi tiết sản phẩm
    \end{itemize}
    \begin{figure}[H]
        \begin{center}
        \includegraphics[scale=0.7]{images/hieu/chap-4/product.png}
        \vspace*{5mm}
        \caption{Phần sản phẩm - Product}
        \end{center}
    \end{figure}
    \item \textbf{Phần Footer}
    \newline
    Phần footer là phần cuối trang web chứa các thông tin về trang web, các liên kết đến các trang mạng xã hội, các liên kết đến các trang khác.
    \begin{figure}[H]
        \begin{center}
        \includegraphics[scale=0.5]{images/hieu/chap-4/footer.png}
        \vspace*{5mm}
        \caption{Phần cuối trang - Footer}
        \end{center}
    \end{figure}
\end{itemize}
\chapter{Xây dựng phiên bản thử nghiệm}
\section{Công nghệ sử dụng}
Để hiện thực hệ thống, nhóm quyết định sử dụng các công nghệ sau:
\begin{itemize}
    \item ReactJS: Hiện thực UI, frontend.
    \item Java Springboot: Hiện thực microservice, backend.
    \item Kubernetes: Deploy các microservice.
    \item Minikube: Chạy Kubernetes cluster trên local.
\end{itemize}

\section{Giới hạn phạm vi}
\noindent Sau khi cân nhắc kỹ lưỡng, để đảm bảo cho phiên bản demo thể hiện được trọn vẹn và đầy đủ nhất các tính chất cốt lõi của hệ thống, nhóm đã giới hạn phạm vi hiện thực của hệ thống xuống còn các thành phần như sau:
\subsection{Về mặt nghiệp vụ}
\noindent Sau khi bàn bạc, nhóm đi tới thống nhất là sẽ hiện thực phần Trang chủ (Home page - Catalog), vì đó là thành phần mà người dùng sẽ gặp đầu tiên khi bắt đầu truy câp vào hệ thống.
\subsection{Về mặt thành phần hệ thống}
\noindent 
% \input{sections/chapter6}
\chapter{Tổng kết}
\section{Kết quả đạt được}
\subsection{Đối với việc tìm hiểu và phân tích nghiệp vụ}
\noindent Căn cứ vào mục tiêu, nhiệm vụ của đề tài đã đề ra, nhóm đã thực hiện được những điều sau:
\begin{itemize}
    \item Tiến hành phân tích các yêu cầu cần có của hệ thống, xác định trọng tâm của đồ án là thực hiện hệ thống thỏa mãn được tính sẵn sàng và tính mở rộng cao.
    \item Xác định các hướng giải quyết bài toán có thể thực hiện được.
    \item Phân tích các yêu cầu chức năng và phi chức năng của hệ thống.
\end{itemize}

\subsection{Đối với cơ sở lý thuyết và công nghệ}
\noindent Dựa vào mục tiêu, yêu cầu nghiệp vụ của đồ án, nhóm đã làm được những việc sau:
\begin{itemize}
    \item Tiến hành so sánh, tham khảo, phân tích các bài viết, ví dụ của các hệ thống tương tự để đưa ra giải pháp phù hợp với điều kiện thực tế.
    \item Tìm hiểu các khái niệm, ý tưởng, kiến trúc, cách hoạt động của hệ thống Kubernetes.
    \item Tìm hiểu được cách các hệ thống microservice được đưa lên Kubernetes.
    \item Chọn ngôn ngữ lập trình, framework, công nghệ phù hợp để xây dụng hệ thống: Frontend (web), các microservice backend và database, công cụ để xây dựng hệ thống Kubernetes ở môi trường local.
\end{itemize}
\subsection{Đối với phân tích và thiết kế hệ thống}
\noindent Dựa vào các thông tin cơ sở lý thuyết, công nghệ đã tìm hiểu, cộng thêm các trải nghiệm thực tế thì nhóm đã đưa ra được thiết kế kiến trúc hoàn chỉnh của hệ thống:
\begin{itemize}
    \item Đưa ra được kiến trúc hoàn chỉnh cho hệ thống, tận dụng các dịch vụ, giải pháp của nền tảng Kubernetes để hoàn thành mục tiêu, nhiệm vụ đề ra trong đồ án.
    \item Tối ưu lại giải pháp cho phù hợp với điều kiện kinh tế hiện tại của mỗi cá nhân.
\end{itemize}
\subsection{Đối với quá trình phát triển ứng dụng}
\noindent Trong suốt giai đoạn Đồ án chuyên ngành, nhóm đã làm việc một cách bài bản, có quy trình, kế hoạch cụ thể và chi tiết.
\begin{itemize}
    \item Nhóm đã thực hiện đầy đủ các giai đoạn: Phân tích, tìm hiểu yêu cầu nghiệp vụ; Thiết kế hệ thống; Triển khai ở quy mô nhỏ.
    \item Công cụ tổ chức và quản lý mã nguồn: Nhóm sử dụng Git và Github để làm tăng hiệu quả công việc.
    \item Về luồng công việc: Nhóm có áp dụng scrum-agile vào việc quản lý, phân chia, theo dõi công việc của các thành viên trong nhóm, tuy nhiên cũng có điều chỉnh cho phù hợp với đặc thù của từng thành viên.
\end{itemize}
\section{Đánh giá giải pháp và kết quả đạt được}
\subsection{Đánh giá thiết kế}
\noindent Hệ thống được xây dựng nhằm mục đích là thử nghiệm các lý thuyết để có được một hệ thống vừa có tính sẵn sàng cao mà vừa có tính mở rộng cao. Đồng thời, giải pháp được áp dụng trong hệ thống có thể sử dụng với nhiều môi trường cloud khác nhau, tăng tính linh động khi triển khai thực tế.
\subsubsection{Ưu điểm}
\begin{itemize}
    \item Hệ thống đảm bảo được tính sẵn sàng cao, tính mở rộng cao.
    \item Phù hợp để chạy trên môi trường cloud, có thể tiết kiệm tối đa chi phí vận hành.
    \item Tương thích với mọi nền tảng cloud có hỗ trợ Kubernetes.
\end{itemize}
\subsubsection{Nhược điểm}
\begin{itemize}
    \item Việc deploy khá phức tạp, cần có sự hiểu biết về Kubernetes và nền tảng cloud dùng để deploy.
    \item Việc kiểm thử cũng tốn nhiều công sức do việc chạy ở local phức tạp và hạn chế hơn khá nhiều khi chạy trên môi trường cloud, tuy nhiên nếu đưa lên cloud sớm thì sẽ không tối ưu về mặt chi phí.
\end{itemize}
\subsection{Đánh giá tính khả thi}
\noindent Để đánh giá tính khả thi của hệ thống trong thực tế, nhóm đã dựa trên 3 tiêu chí là Tính khả thi về mặt Công nghệ, Tính khả thi về mặt Kinh tế, Tính khả thi về mặt Vận hành.
\subsubsection{Tính khả thi về mặt Công nghệ}
\noindent Hệ thống sử dụng các tool, framework nổi tiếng, được cộng đồng hỗ trợ nhiệt tình nên việc bảo trì, bảo dưỡng, sửa chữa lỗi sẽ dễ dàng hơn nhiều.
\begin{itemize}
    \item Frontend: ReactJS
    \item Backend: Java Springboot, RabbitMQ
    \item Database: Postgres
    \item Deploy: Kubernetes, Docker, Minikube.
\end{itemize}
\subsubsection{Tính khả thi về mặt Kinh tế}
\noindent Các công nghệ được dùng đều là mã nguồn mở, miễn phí, do đó sẽ không tốn chi phí bản quyền.\\[0.5cm]
Về việc triển khai hệ thống, nhóm sẽ ưu tiên kiểm thử ở local trước khi đưa lên cloud, do đó cũng không cần tốn nhiều chi phí vận hành cho các nhà cung cấp dịch vụ trong giai đoạn phát triển hệ thống.
\subsubsection{Tính khả thi về mặt Vận hành}
\noindent Về phía người dùng, hệ thống không có gì khác biệt so với các hệ thống thương mại điện tử khác, nên không cần làm quen bất kỳ điều gì mới. Về phía doanh nghiệp, họ cần có kỹ sư DevOp để quản lý và bảo trì hệ thống, nhưng khối lượng công việc cũng không nặng nếu như họ đã có sẵn các luồng CI/CD tự động cho việc nâng cấp và cập nhật hệ thống.
\subsection{Đánh giá lợi ích}
\noindent Về mặt thực tiễn, hệ thống có thể xem như một bản mẫu cho tổ chức, doanh nghiệp muốn xây dựng một hệ thống thương mại điện tử đáng tin cậy, đảm bảo luôn sẵn sàng phục vụ khách hàng, cũng như dễ dàng thêm và mở rộng tính năng mới.
\subsection{Đánh giá kết quả đạt được}
\subsubsection{Ưu điểm}
\begin{itemize}
    \item Đáp ứng được các yêu cầu cơ bản của một hệ thống thương mại điện tử.
    \item Giải quyết được bài toán về tính sẵn sàng cao và tính mở rộng cao.
    \item Sử dụng Message Queue để giao tiếp giữa các microservices một cách hiệu quả, đặc biệt kết hợp với kiến trúc microservice giúp tăng tính decoupling và tính mở rộng (về tính năng) cho hệ thống sau này
\end{itemize}
\subsubsection{Nhược điểm}
\begin{itemize}
    \item Kiến trúc khá phức tạp, tốn nhiều thời gian, công sức để triển khai.
    \item Chưa có luồng CI/CD để tự động hóa, hỗ trợ phát triển hệ thống.
    \item Chức năng Autoscaling còn chưa được tùy chỉnh hợp lý.
\end{itemize}
\section{Hướng phát triển đề tài trong tương lai}
\noindent Trải qua quá trình nghiên cứu lý thuyết, tổng hợp đưa ra giải pháp và xây dựng phiên bản thử nghiệm, nhóm nhận thấy trong tương lai đề tài còn cần thực hiện thêm các công việc như:
\begin{itemize}
    \item Triển khai trên môi trường cloud để có đánh giá chính xác hơn.
    \item Xây dựng luồng CI/CD để tăng hiệu quả công việc.
    \item Xây dựng tính năng gợi ý sản phầm phù hợp với lich sử mua hàng của từng cá nhân.
    \item Nghiên cứu tận dụng thêm các plugin của Kubernetes để có thể scale hệ thống hiệu quả hơn.
    \item Nghiên cứu thêm về cách đo tính sẵn sàng của hệ thống.
\end{itemize}

% \input{sections/conclusion}
\cleardoublepage
\phantomsection
\addcontentsline{toc}{chapter}{Tài liệu tham khảo}

\begin{thebibliography}{1}
    \bibitem {react}
Meta Platforms Inc (2022), \emph{React Tutorial}, URL: \url{https://reactjs.org/}, truy cập lần cuối: 03/10/2022. 


\end{thebibliography}
% \input{sections/appendixes}
% \input{sections/task_division}

\endgroup



\end{document}
\chapter{Phân tích yêu cầu}
\section{Phân tích yêu cầu nghiệp vụ}
\subsection{Yêu cầu chức năng}
Hệ thống có một số tính năng chính sau:
\begin {list} {-}{}
    \item Người dùng có thể đăng ký tài khoản cho riêng mình.
    \item Người dùng có thể đăng nhập vào hệ thống với tài khoản và mật khẩu đã đăng ký thành công trước đó.
    \item Người dùng có thể đổi mật khẩu khi cần thiết.
    \item Khi truy cập và hệ thống, người dùng có thể xem được tất cả các sản phẩm của cửa hàng.
    \item Hệ thống có thể phân loại sản phẩm theo từng loại khác nhau.
    \item Người dùng có thể tìm kiếm sản phẩm theo tên, loại, giá, kích thước\dots
    \item Người dùng có thể xem chi tiết sản phẩm để lựa chọn được sản phẩm phụ hợp nhất.
    \item Người dùng có thể thêm sản phẩm vào giỏ hàng.
    \item Người dùng có thể xem giỏ hàng của mình.
    \item Người dùng có thể xóa sản phẩm, điều chỉnh số lượng sản phẩm trong giỏ hàng.
    \item Người dùng có thể xem tổng số tiền hiện tại của giỏ hàng.
    \item Người dùng có thể thanh toán những sản phẩm trong giỏ hàng bằng nhiều hình thức khác nhau như: tiền mặt, thẻ tín dụng, ví điện tử\dots
    \item Người dùng có thể xem lịch sử mua hàng của mình.
    \item Người dùng có thể để lại những đánh giá về sản phẩm mà mình đã mua.
    \item Hệ thống phân quyền người dùng theo từng vai trò khác nhau gồm có: khách hàng (customer), nhân viên (staff), quản lý (manager).
    \item Khi đăng nhập với vai trò là nhân viên, người dùng có thể thêm, sửa, xóa, điều chỉnh sản phẩm và quản lý đơn hàng.
    \item Khi đăng nhập với vai trò là quản lý, người dùng có thể thêm, sửa, xóa, điều chỉnh sản phẩm, quản lý đơn hàng, và quản lý người dùng sử dụng hệ thống (thêm, sửa, xóa, chặn\dots).
    \item Hệ thống thống kê doanh thu thành biểu đồ theo từng ngày,tháng,năm để nhân viên và quản lý có thể dễ dàng theo dõi.
\end {list}
\subsection{Yêu cầu phi chức năng}
Hệ thống có một số yêu cầu phi chức năng như sau:
\begin {list} {-}{}
    \item Hệ thống có giao diện thân thiện với người dùng.
    \item Hệ thống có thể dễ dàng sử dụng đối với người dùng mới.
    \item Hệ thống có thể hoạt động tốt trên hầu hết các trình duyệt hiện nay như Chrome, Firefox, Safari, Microsoft Edge\dots
    \item Hệ thống hoạt động tốt trên nhiều loại thiết bị khác nhau như máy tính, điện thoại, máy tính bảng\dots
    \item Hệ thống có thể chạy mượt mà, ổn định.
    \item Hệ thống có thể chịu tải tốt, phục vụ nhiều người dùng cùng một lúc.
    \item Hệ thống phục vụ người dùng 24/7.
    \item Hệ thống có thể bảo mật thông tin người dùng.
\newpage
\section{Phân tích hệ thống}
\subsection{Use case diagram}
\subsubsection{Use case diagram cho toàn bộ tính năng của hệ thống}
\begin{figure}[h]
    \centering
    \includegraphics[scale=0.4]{images/hieu/chap-3/usecase-diagram.png}
    \caption{Use case diagram cho toàn bộ tính năng của hệ thống}
\end{figure}
\newpage
\subsubsection{Đặc tả use case cho một số tính năng chính}
\begin{itemize}
    \item Đăng nhập
    \begin{table}[h]
        \begin{tabular}{|l|l|}
        \hline
        \textbf{Use-case name}    & \textbf{Đăng nhập - Login}                                                                                                                                                                                                                                                                                                                                                                                                                           \\ \hline
        \textbf{Actor}            & User (Manager, Customer, Staff)                                                                                                                                                                                                                                                                                                                                                                                                                      \\ \hline
        \textbf{Description}      & Người dùng đăng nhập vào hệ thống                                                                                                                                                                                                                                                                                                                                                                                                                    \\ \hline
        \textbf{Pre-condition}    & Người dùng đã đăng ký tài khoản thàng công                                                                                                                                                                                                                                                                                                                                                                                                           \\ \hline
        \textbf{Post-condition}   & Người dùng đăng nhập thành công                                                                                                                                                                                                                                                                                                                                                                                                                      \\ \hline
        \textbf{Normal-flow}      & \begin{tabular}[c]{@{}l@{}}1.Người dùng chọn biểu tượng đăng nhập trên góc phải màn hình.\\ 2.Hệ thống hiển thị trang đăng nhập.\\ 3.Người dùng nhập thông tin đã đăng ký gồm tên người dùng (username) và \\ mật khẩu (password)\\ 4.Người dùng chọn nút "ĐĂNG NHẬP".\\ 5.1.Hệ thống chuyển đến "TRANG CHỦ" nếu xác thực thông tin thành công.\\ 5.2.Nếu thông tin sai, hệ thống yêu cầu người dùng nhập lại.\\ 6.Đăng nhập thành công\end{tabular} \\ \hline
        \textbf{Alternative-flow} & Không có                                                                                                                                                                                                                                                                                                                                                                                                                                             \\ \hline
        \textbf{Exception}        & \begin{tabular}[c]{@{}l@{}}Nếu người dùng bị chặn bởi người quản lý (Manager) sẽ không thể đăng nhập \\ vào hệ thống\end{tabular}                                                                                                                                                                                                                                                                                                                    \\ \hline
        \end{tabular}
        \begin{center}
            Bảng 3.1: Đặc tả use case đăng nhập
        \end{center}
        \end{table}
        
    \item Đăng ký tài khoản
    \begin{table}[h]
        \begin{tabular}{|l|l|}
        \hline
        \textbf{Use-case name}    & \textbf{Đăng ký - SignUp}                                                                                                                                                                                                                                                                                                                                                                                                                                                                                                                              \\ \hline
        \textbf{Actor}            & User (Manager, Customer, Staff)                                                                                                                                                                                                                                                                                                                                                                                                                                                                                                                        \\ \hline
        \textbf{Description}      & Người dùng đăng ký tài khoản để truy cập vào hệ thống                                                                                                                                                                                                                                                                                                                                                                                                                                                                                                  \\ \hline
        \textbf{Pre-condition}    & Người dùng truy cập vào website thành công                                                                                                                                                                                                                                                                                                                                                                                                                                                                                                             \\ \hline
        \textbf{Post-condition}   & Người dùng đăng ký tài khoản thành công                                                                                                                                                                                                                                                                                                                                                                                                                                                                                                                \\ \hline
        \textbf{Normal-flow}      & \begin{tabular}[c]{@{}l@{}}1.Người dùng chọn biểu tượng đăng nhập trên góc phải màn hình.\\ 2.Hệ thống hiển thị trang đăng nhập.\\ 3.Người dùng nhấn vào liên kết "Bạn chưa có tài khoản?" \\ 4.Hệ thống chuyển sang trang đăng ký tài khoản.\\ 5.Người dùng nhập đầy đủ thông tin mà hệ thống yêu cầu như họ tên, email,\\ địa chỉ, số điện thoại...\\ 6.Chọn nút "ĐĂNG KÝ"\\ 7.1.Hệ thống hiển thị đăng ký thành công \\ 7.2.Hệ thống yêu cầu nhập lại thông tin nếu có sai sót (thiếu trường thông tin,\\ email không thể xác thực...) \\8.Hệ thống lưu thông tin người dùng.\end{tabular} \\ \hline
        \textbf{Alternative-flow} & Không có                                                                                                                                                                                                                                                                                                                                                                                                                                                                                                                                               \\ \hline
        \textbf{Exception}        & Một địa chỉ email chỉ đăng ký được một tài khoản.                                                                                                                                                                                                                                                                                                                                                                                                                                                                                                      \\ \hline
        \end{tabular}
        \begin{center}
            Bảng 3.2: Đặc tả use case đăng ký tài khoản
        \end{center}
    \end{table}
    \newpage
    \item Xem sản phẩm
    \begin{table}[h]
        \begin{tabular}{|l|l|}
        \hline
        \textbf{Use-case name}    & \textbf{Xem sản phẩm - View Products}                                                                                                                                                                                                                                                                                                                                                                                                                                                                                                                                                                                                                                                                                                                                                                           \\ \hline
        \textbf{Actor}            & User (Manager, Customer, Staff)                                                                                                                                                                                                                                                                                                                                                                                                                                                                                                                                                                                                                                                                                                                                                                                 \\ \hline
        \textbf{Description}      & Người dùng xem những sản phẩm được bày bán trên cửa hàng                                                                                                                                                                                                                                                                                                                                                                                                                                                                                                                                                                                                                                                                                                                                                        \\ \hline
        \textbf{Pre-condition}    & Người dùng truy cập vào website thành công                                                                                                                                                                                                                                                                                                                                                                                                                                                                                                                                                                                                                                                                                                                                                                      \\ \hline
        \textbf{Post-condition}   & Người dùng có thể xem sản phẩm của cửa hàng                                                                                                                                                                                                                                                                                                                                                                                                                                                                                                                                                                                                                                                                                                                                                                     \\ \hline
        \textbf{Normal-flow}      & \begin{tabular}[c]{@{}l@{}}1.Người dùng truy cập vào website.\\ 2.Hệ thống tự động chuyển đến trang chủ (HomePage).\\ 3.Tại trang chủ, người dùng có thể lướt để xem tất cả những sản phẩm được   \\ hiển thị theo loại, ngoài ra còn có những sản phẩm nổi bật, những sản phẩm\\ sắp được ra mắt trong tương lai.\\ 4.Người dùng có thể tìm kiếm sản phẩm mình muốn bằng cách nhập từ khoá \\ (tên, loại, giá, kích thước...) vào ô tìm kiếm rồi nhấn nút "TÌM KIẾM".\\ 5.Hệ thống hiển thị kết quả tìm kiếm.\\ 6.Người dùng nhấn vào từng sản phẩm \\ 7.Hệ thống hiển thị thông tin chi tiết của sản phẩm tương ứng.\\ 8.Nếu người dùng muốn mua sản phẩm:\\       8.1.Nhấn vào nút "MUA HÀNG" để trực tiếp mua sản phẩm.\\       8.2.Nhấn vào biểu tưởng giỏ hàng để thêm sản phẩm vào giỏ hàng.\end{tabular} \\ \hline
        \textbf{Alternative-flow} & Không có                                                                                                                                                                                                                                                                                                                                                                                                                                                                                                                                                                                                                                                                                                                                                                                                        \\ \hline
        \textbf{Exception}        & Người dùng phải đăng nhập mới có thể mua hàng hoặc thêm vào giỏ hàng.                                                                                                                                                                                                                                                                                                                                                                                                                                                                                                                                                                                                                                                                                                                                                                                                        \\ \hline
        \end{tabular}
        \begin{center}
            Bảng 3.3: Đặc tả use case xem sản phẩm
        \end{center}
        \end{table}
        
        \item Quản lý giỏ hàng
        \begin{table}[h]
            \begin{tabular}{|l|l|}
            \hline
            \textbf{Use-case name}    & \textbf{Quản lý giỏ hàng - Cart Manager}                                                                                                                                                                                                                                 \\ \hline
            \textbf{Actor}            & User (Manager, Customer, Staff)                                                                                                                                                                                                                                          \\ \hline
            \textbf{Description}      & Người dùng quản lý giỏ hàng của mình                                                                                                                                                                                                                                     \\ \hline
            \textbf{Pre-condition}    & Người dùng đăng nhập vào hệ thống thành công                                                                                                                                                                                                                             \\ \hline
            \textbf{Post-condition}   & Người dùng quản lý giỏ hàng thành công                                                                                                                                                                                                                                   \\ \hline
            \textbf{Normal-flow}      & \begin{tabular}[c]{@{}l@{}}1.Người dùng nhấn vào biểu tượng giỏ hàng nằm trên thanh điều hướng.\\ 2.Hệ thống hiển thị chi tiết trang giỏ hàng.\\ 3.Tại đây, người dùng có thể.\\ -Thêm / Xoá sản phẩm khỏi giỏ hàng.\\ -Tăng / giảm số lượng từng sản phẩm.\end{tabular} \\ \hline
            \textbf{Alternative-flow} & Không có                                                                                                                                                                                                                                                                 \\ \hline
            \textbf{Exception}        & Không có                                                                                                                                                                                                                                                                 \\ \hline
            \end{tabular}
            \begin{center}
                Bảng 3.4: Đặc tả use case quản lý giỏ hàng
            \end{center}
            \end{table}
        \newpage
        \item Thanh toán
            \begin{table}[h]
                \begin{tabular}{|l|l|}
                \hline
                \textbf{Use-case name}    & \textbf{Thanh toán - Payment}                                                                                                                                                                                                                                                                                                                                                                                                                                                                                                                                                                                                                                                                                                                                                                                                              \\ \hline
                \textbf{Actor}            & User (Manager, Customer, Staff), Credit Payment Service                                                                                                                                                                                                                                                                                                                                                                                                                                                                                                                                                                                                                                                                                                                                                                                    \\ \hline
                \textbf{Description}      & Người dùng thanh toán đơn hàng của mình                                                                                                                                                                                                                                                                                                                                                                                                                                                                                                                                                                                                                                                                                                                                                                                                    \\ \hline
                \textbf{Pre-condition}    & Người dùng đăng nhập vào hệ thống thành công                                                                                                                                                                                                                                                                                                                                                                                                                                                                                                                                                                                                                                                                                                                                                                                               \\ \hline
                \textbf{Post-condition}   & Người dùng thanh toán thành công                                                                                                                                                                                                                                                                                                                                                                                                                                                                                                                                                                                                                                                                                                                                                                                                           \\ \hline
                \textbf{Normal-flow}      & \begin{tabular}[c]{@{}l@{}}1.Người dùng nhấn vào biểu tượng giỏ hàng nằm trên thanh điều hướng.\\ 2.Hệ thống hiển thị chi tiết trang giỏ hàng.\\ 3.Tại giỏi hàng, người dùng có thể chọn lại những sản phẩm cần mua ngay.\\ 4.Chọn nút "THANH TOÁN".\\ 5.Hệ thống tính toán tổng số tiền cần trả (bao gồm cả phí vận chuyển) rồi\\ hiển thị lên màn hình.\\ 6.Người dùng lựa chọn phương thức thanh toán:\\ 6.1.Thanh toán tiền mặt: Người dùng để lại thông tin, đơn hàng sẽ được\\ tổng hợp và người dùng có thể ra cửa hàng để thanh toán.\\ 6.2.Thanh toán bẳng thẻ tín dụng, thẻ ngân hàng, ví điện tử:\\ -Người dùng chọn dịch vụ phù hợp.\\ -Dịch vụ bên thứ 3 sẽ liên kết và xử lý thanh toán.\\ -6.2.1.Hệ thống thông báo thanh toán thành công.\\ -6.2.2.Hệ thống thông báo lỗi, yêu cầu người dùng thực hiện lại.\end{tabular} \\ \hline
                \textbf{Alternative-flow} & Không có                                                                                                                                                                                                                                                                                                                                                                                                                                                                                                                                                                                                                                                                                                                                                                                                                                   \\ \hline
                \textbf{Exception}        & Không có                                                                                                                                                                                                                                                                                                                                                                                                                                                                                                                                                                                                                                                                                                                                                                                                                                   \\ \hline
                \end{tabular}
                \begin{center}
                    Bảng 3.5: Đặc tả use case thanh toán
                \end{center}
                \end{table}
                
        \item Quản lý sản phẩm
            \begin{table}[h]
                \begin{tabular}{|l|l|}
                \hline
                \textbf{Use-case name}    & \textbf{Quản lý sản phẩm - Products Manager}                                                                                                                                                                                                                                                                                                                                                                                                                                                                                                                                                                                                                                        \\ \hline
                \textbf{Actor}            & Manager, Staff                                                                                                                                                                                                                                                                                                                                                                                                                                                                                                                                                                                                                                                                      \\ \hline
                \textbf{Description}      & Người quản lý và nhân viên quản lý sản phẩm của cửa hàng                                                                                                                                                                                                                                                                                                                                                                                                                                                                                                                                                                                                                            \\ \hline
                \textbf{Pre-condition}    & Người dùng đăng nhập vào hệ thống với vai trò là quản lý hoặc nhân viên                                                                                                                                                                                                                                                                                                                                                                                                                                                                                                                                                                                                             \\ \hline
                \textbf{Post-condition}   & Người dùng quản lý sản phẩm thành công.                                                                                                                                                                                                                                                                                                                                                                                                                                                                                                                                                                                                                                             \\ \hline
                \textbf{Normal-flow}      & \begin{tabular}[c]{@{}l@{}}1.Tại trang web, người dùng chọn nút "SẢN PHẨM".\\ 2.Hệ thống chuyển đến trang quản lý sản phẩm.\\ 3.Tại đây, người dùng có thể thấy tất cả các sản phẩm hiện có cũng như những\\ thông tin chi tiết của chúng.\\ 4.1.Người dùng có thể nhấn "CHỈNH SỬA" để điều chỉnh thông tin sản phẩm\\ 4.2.Người dùng có thể nhấn "THÊM" để thêm sản phẩm mới\\ -4.2.1.Hệ thống hiện thị một popup để người dùng thêm sản phẩm mới\\ -4.2.2.Người dùng nhập thông tin cần thiết cho sản phẩm mới\\ -4.2.3.Người dùng chọn "THÊM" để xác nhận \\ 4.3.Người dùng chọn "XOÁ" để xoá sản phẩm khỏi hệ thống\\ 5.Hệ thống lưu lại hành động của người dùng.\end{tabular} \\ \hline
                \textbf{Alternative-flow} & Không có                                                                                                                                                                                                                                                                                                                                                                                                                                                                                                                                                                                                                                                                            \\ \hline
                \textbf{Exception}        & Không có                                                                                                                                                                                                                                                                                                                                                                                                                                                                                                                                                                                                                                                                            \\ \hline
                \end{tabular}
                \begin{center}
                    Bảng 3.6: Đặc tả use case quản lý sản phẩm
                \end{center}
                \end{table}
            \newpage
            \item Quản lý người dùng
            \begin{table}[h]
            \begin{tabular}{|l|l|}
            \hline
            \textbf{Use-case name}    & \textbf{Quản lý người dùng - User Manager}                                                                                                                                                                                                                                                                                                                                                                                                                                                                                                                                                                                              \\ \hline
            \textbf{Actor}            & Manager                                                                                                                                                                                                                                                                                                                                                                                                                                                                                                                                                                                                                                 \\ \hline
            \textbf{Description}      & Người quản lý quản lý những người sử dụng hệ thống                                                                                                                                                                                                                                                                                                                                                                                                                                                                                                                                                                                      \\ \hline
            \textbf{Pre-condition}    & Người dùng đăng nhập vào hệ thống với vai trò là quản lý                                                                                                                                                                                                                                                                                                                                                                                                                                                                                                                                                                                \\ \hline
            \textbf{Post-condition}   & Manager quản lý người dùng thành công                                                                                                                                                                                                                                                                                                                                                                                                                                                                                                                                                                                                   \\ \hline
            \textbf{Normal-flow}      & \begin{tabular}[c]{@{}l@{}}1.Hệ thống chuyển sang trang quản lý.\\ 2.Manager chọn "NGƯỜI DÙNG"\\ 3.Hệ thống hiển thị tất cả người dùng của trang web.\\ 4.Tại đây, manager có thể:\\ 4.1.Thêm người dùng mới vào hệ thống bằng cách chọn nút "THÊM"\\ 4.2.Xoá người dùng khỏi hệ thống bằng cách chọn nút "XOÁ" nằm bên cạnh \\ người dùng cần xoá.\\ 4.3.Điều chỉnh thông tin người dùng bằng cách chọn nút "ĐIỀU CHỈNH"\\ 4.4.Chặn không cho người dùng tiếp tục sử dụng trang web bằng cách chọn \\ nút "CHẶN".\\ 4.5.Bỏ chặn người dùng bằng cách chọn nút "BỎ CHẶN"\\ 5.Hệ thống lưu lại hành động của người quản lý.\end{tabular} \\ \hline
            \textbf{Alternative-flow} & Không có                                                                                                                                                                                                                                                                                                                                                                                                                                                                                                                                                                                                                                \\ \hline
            \textbf{Exception}        & Không có                                                                                                                                                                                                                                                                                                                                                                                                                                                                                                                                                                                                                                \\ \hline
            \end{tabular}
            \begin{center}
                Bảng 3.7: Đặc tả use case quản lý người dùng
            \end{center}
            \end{table}
            \item Xem thống kê doanh thu
            \begin{table}[h]
                \begin{tabular}{|l|l|}
                \hline
                \textbf{Use-case name}    & \textbf{Xem thống kê doanh thu - View Statistic}                                                                                                                                                                                                                                                                                                                                 \\ \hline
                \textbf{Actor}            & Manager                                                                                                                                                                                                                                                                                                                                                                          \\ \hline
                \textbf{Description}      & Người quản lý có thể xem doanh thu của cửa hàng                                                                                                                                                                                                                                                                                                                                  \\ \hline
                \textbf{Pre-condition}    & Người dùng đăng nhập vào hệ thống với vai trò là quản lý                                                                                                                                                                                                                                                                                                                         \\ \hline
                \textbf{Post-condition}   & Manager xem doanh thu thành công                                                                                                                                                                                                                                                                                                                                                 \\ \hline
                \textbf{Normal-flow}      & \begin{tabular}[c]{@{}l@{}}1.Hệ thống chuyển sang trang quản lý.\\ 2.Manager chọn nút "THỐNG KÊ"\\ 3.Hệ thống chuyển sang trang thống kê\\ 4.Tại trang thống kê, người quản lý chọn khoảng thời gian và chọn sản phẩm \\ \\ (có thể xem thống kê của một hoặc nhiều loại sản phẩm)\\ 5.Hệ thống hiển thi thống kê dưới dạng các biểu đồ.\\ 6.Hoàn tất xem thống kê.\end{tabular} \\ \hline
                \textbf{Alternative-flow} & Không có                                                                                                                                                                                                                                                                                                                                                                         \\ \hline
                \textbf{Exception}        & Không có                                                                                                                                                                                                                                                                                                                                                                         \\ \hline
                \end{tabular}
                \begin{center}
                    Bảng 3.8: Đặc tả use case xem thống kê doanh thu
                \end{center}
                \end{table}
    \end{itemize}
\newpage
\subsection{Activity diagram}
\subsubsection{Activity diagram cho user là khách hàng}
\begin{figure}[h]
    \centering
    \includegraphics[scale=0.45]{images/hieu/chap-3/user-activity-diagram.png}
    \caption{Activity diagram cho user là khách hàng}
\end{figure}

\newpage
\subsubsection{Activity diagram cho user là quản lý}
\begin{figure}[h]
    \centering
    \includegraphics[scale=0.4]{images/hieu/chap-3/admin-activity-diagram.png}
    \caption{Activity diagram cho user là quản lý}
\end{figure}
\newpage
\subsubsection{Activity diagram cho chức năng đăng nhập - đăng ký}
\begin{figure}[h]
    \centering
    \includegraphics[scale=0.6]{images/hieu/chap-3/login-signup-activity-diagram.png}
    \caption{Activity diagram cho chức năng đăng nhập - đăng ký}
\end{figure}
\newpage
\subsubsection{Activity diagram cho chức năng quản lý}
\begin{figure}[h]
    \centering
    \includegraphics[scale=0.5]{images/hieu/chap-3/manage-activity-diagram.png}
    \caption{Activity diagram cho chức năng quản lý}
\end{figure}
\newpage
\subsubsection{Activity diagram cho chức năng mua sắm}
\begin{figure}[h]
    \centering
    \includegraphics[scale=0.4]{images/hieu/chap-3/shopping-activity-diagram.png}
    \caption{Activity diagram cho chức năng mua sắm}
\end{figure}
\newpage
\begin{figure}[h]
    \centering
    \includegraphics[scale=0.6]{images/hieu/chap-3/Statistic-activity-diagram.png}
    \caption{Activity diagram cho chức năng xem thống kê}
\end{figure}
\newpage
\subsection{Sequence diagram}
\subsubsection{Sequence diagram cho chức năng đăng nhập - đăng ký}
\begin{figure}[h]
    \centering
    \includegraphics[scale=0.6]{images/hieu/chap-3/login-signup-sequence-diagram.png}
    \caption{Sequence diagram cho chức năng đăng nhập - đăng ký}
\end{figure}
Trong đó:
\begin{itemize}
    \item User: người dùng (khách hàng, quản lý, nhân viên) đăng ký tài khoản và đăng nhập vào hệ thống.
    \item UserInterface: giao diện trang đăng nhập và trang đăng ký tương tác với người dùng.
    \item RigisterController: khối xử lý chức năng đăng ký tài khoản của người dùng.
    \item LoginController: khối xử lý chức năng đăng nhập của người dùng.
    \item AccountDatabase: cơ sở dữ liệu lưu thông tin tài khoản của người dùng.
\end{itemize}
\newpage
\subsubsection{Sequence diagram cho chức năng mua sắm}
\begin{figure}[h]
    \centering
    \includegraphics[scale=0.5]{images/hieu/chap-3/shopping-sequence-diagram.png}
    \caption{Sequence diagram cho chức năng mua sắm}
\end{figure}
Trong đó:
\begin{itemize}
    \item UserInterface: giao diện trang chủ, trang chi tiết sản phẩm, trang giỏ hàng và trang thanh toán tương tác với người dùng.
    \item Product Database: cơ sở dữ liệu lưu thông tin sản phẩm của cửa hàng.
    \item Cart Controller: khối xử lý chức năng quản lý giỏ hàng của người dùng.
    \item Payment Controller: khối xử lý chức năng thanh toán của người dùng.
    \item Payment Serice: dịch vụ thanh toán bên thứ 3.
    \item Order Database: cơ sở dữ liệu lưu thông tin đơn hàng của cửa hàng.
\end{itemize}
\newpage
\subsubsection{Sequence diagram cho chức năng quản lý sản phẩm}
\begin{figure}[h]
    \centering
    \includegraphics[scale=0.5]{images/hieu/chap-3/manage-product-sequence-diagram.png}
    \caption{Sequence diagram cho chức năng quản lý sản phẩm}
\end{figure}
Trong đó:
\begin{itemize}
    \item Manager: là người quản lý, sẽ quản lý tất cả các sản phẩm của cửa hàng.
    \item LoginInterface: giao diện trang đăng nhập tương tác với người dùng.
    \item Authorization: khối xử lý chức năng xác thực người dùng có phải là người quản lý hay không.
    \item Manager Interface: giao diện trang quản lý sản phẩm tương tác với người dùng.
    \item Product Controller: khối xử lý chức năng quản lý sản phẩm của người dùng.
    \item Product Database: cơ sở dữ liệu lưu thông tin sản phẩm của cửa hàng.
    \item Customer Interface: giao diện trang chủ tương tác với người dùng là khác hàng (trong trường hợp người dùng không phải là người quản lý).
\end{itemize}
\newpage
\subsubsection{Sequence diagram cho chức năng quản lý tài khoản}
\begin{figure}[h]
    \centering
    \includegraphics[scale=0.5]{images/hieu/chap-3/manage-account-sequence-diagram.png}
    \caption{Sequence diagram cho chức năng quản lý tài khoản}
\end{figure}
Trong đó:
\begin{itemize}
    \item Manager: là người quản lý, sẽ quản lý tất cả các sản phẩm của cửa hàng.
    \item LoginInterface: giao diện trang đăng nhập tương tác với người dùng.
    \item Authorization: khối xử lý chức năng xác thực người dùng có phải là người quản lý hay không.
    \item Manager Interface: giao diện trang quản lý tài khoản tương tác với người dùng.
    \item Account Controller: khối xử lý chức năng quản lý tài khoản của người dùng.
    \item Account Database: cơ sở dữ liệu lưu thông tin tài khoản của cửa hàng.
    \item Customer Interface: giao diện trang chủ tương tác với người dùng là khác hàng (trong trường hợp người dùng không phải là người quản lý).
\end{itemize}
\newpage
\subsubsection{Sequence diagram cho chức năng quản lý bài viết}
\begin{figure}[h]
    \centering
    \includegraphics[scale=0.5]{images/hieu/chap-3/manage-blog-sequence-diagram.png}
    \caption{Sequence diagram cho chức năng quản lý bài viết}
\end{figure}
Trong đó:
\begin{itemize}
    \item Manager: là người quản lý, sẽ quản lý tất cả các sản phẩm của cửa hàng.
    \item LoginInterface: giao diện trang đăng nhập tương tác với người dùng.
    \item Authorization: khối xử lý chức năng xác thực người dùng có phải là người quản lý hay không.
    \item Manager Interface: giao diện trang quản lý bài viết tương tác với người dùng.
    \item Blog Controller: khối xử lý chức năng quản lý bài viết của người dùng.
    \item Blog Database: cơ sở dữ liệu lưu thông tin bài viết của cửa hàng.
    \item Customer Interface: giao diện trang chủ tương tác với người dùng là khác hàng (trong trường hợp người dùng không phải là người quản lý).
\end{itemize}
\newpage
\subsubsection{Sequence diagram cho chức năng xem thống kê doanh thu}
\begin{figure}[h]
    \centering
    \includegraphics[scale=0.5]{images/hieu/chap-3/statistic-sequence-diagram.png}
    \caption{Sequence diagram cho chức năng xem thống kê doanh thu}
\end{figure}
Trong đó:
\begin{itemize}
    \item Manager: là người quản lý, sẽ quản lý tất cả các sản phẩm của cửa hàng.
    \item LoginInterface: giao diện trang đăng nhập tương tác với người dùng.
    \item Authorization: khối xử lý chức năng xác thực người dùng có phải là người quản lý hay không.
    \item Manager Interface: giao diện trang quản lý doanh thu tương tác với người dùng.
    \item Statistic Controller: khối xử lý chức năng xem thống kê doanh thu của người dùng.
    \item Order Database: cơ sở dữ liệu lưu thông tin đơn hàng của cửa hàng.
    \item Customer Interface: giao diện trang chủ tương tác với người dùng là khác hàng (trong trường hợp người dùng không phải là người quản lý).
\end{itemize}
\newpage
\subsection{Database diagram}
\subsubsection {Entity Relationship Diagram - ERD}
Sơ đồ ERD đá tinh giản hoá các thuộc tính của các thực thể
\newline
\textbf{Ghi chú:} Đây chỉ là sơ đồ cho chức năng liên quan đến sản phẩm
\begin{figure}[h]
    \centering
    \includegraphics[scale=0.6]{images/hieu/chap-3/database-diagram.png}
    \caption{Entity Relationship Diagram - ERD}
\end{figure}
\subsubsection {Mô tả chi tiết thực thể}
\begin{itemize}
    \item \textbf{Product:} Thực thể này lưu trữ thông tin về sản phẩm của cửa hàng.
        \begin{table}[h]
        \begin{tabular}{|l|l|l|}
        \hline
        \textbf{Thuộc tính} & \textbf{Kiểu dữ liệu} & \textbf{Mô tả}                            \\ \hline
        ID                  & int                   & Mã sản phẩm - khoá chính                  \\ \hline
        name                & varchar               & Tên sản phẩm                              \\ \hline
        desc                & text                  & Mô tả sản phẩm                            \\ \hline
        SKU                 & varchar               & Kết hợp giữa mã sản phẩm và loại sản phẩm \\ \hline
        category\_id        & Long                  & Mã loại sản phẩm - khoá ngoại             \\ \hline
        inventory\_id       & Long                  & Mã kho lưu trữ sản phẩm - khoá ngoại      \\ \hline
        price               & decimal               & Giá của sản phẩm                          \\ \hline
        discount\_id        & int                   & Mã khuyến mãi của sản phẩm - khoá ngoại   \\ \hline
        created\_at         & timestamp             & Thời gian tạo                             \\ \hline
        modified\_at        & timestamp             & Thời gian điều chỉnh                      \\ \hline
        image\_link         & string                & Liên kết chứa hỉnh ảnh sản phẩm           \\ \hline
        \end{tabular}
        \end{table}
    \item \textbf{Product Category:} Thực thể này lưu trữ thông tin về loại sản phẩm của cửa hàng.
            \begin{table}[h]
                \begin{tabular}{|l|l|l|}
                \hline
                \textbf{Thuộc tính} & \textbf{Kiểu dữ liệu} & \textbf{Mô tả}                \\ \hline
                ID                  & int                   & Mã loại sản phẩm - khoá chính \\ \hline
                name                & varchar               & Tên loại sản phẩm             \\ \hline
                desc                & text                  & Mô tả loại sản phẩm           \\ \hline
                deleted\_at         & timestamp             & Thời giản xoá loại sản phẩm   \\ \hline
                modified\_at        & timestamp             & Thời gian điều chỉnh          \\ \hline
                created\_at         & timestamp             & Thời gian tạo                 \\ \hline
                \end{tabular}
                \end{table}
    \item \textbf{Product Inventory:} Thực thể này lưu trữ thông tin về kho lưu trữ sản phẩm của cửa hàng.
        \begin{table}[h]
        \begin{tabular}{|l|l|l|}
        \hline
        \textbf{Thuộc tính} & \textbf{Kiểu dữ liệu} & \textbf{Mô tả}                \\ \hline
        ID                  & int                   & Mã loại sản phẩm - khoá chính \\ \hline
        quantity            & int                   & Số lượng sản phẩm             \\ \hline
        modified\_at        & timestamp             & Thời gian điều chỉnh          \\ \hline
        created\_at         & timestamp             & Thời gian tạo                 \\ \hline
        \end{tabular}
        \end{table}
    \item \textbf{Discount:} Thực thể này lưu trữ thông tin về khuyến mãi của sản phẩm.
        \begin{table}[h]
        \begin{tabular}{|l|l|l|}
        \hline
        \textbf{Thuộc tính} & \textbf{Kiểu dữ liệu} & \textbf{Mô tả}           \\ \hline
        ID                  & int                   & Mã giảm giá - khoá chính \\ \hline
        name                & varchar               & Tên giảm giá             \\ \hline
        desc                & text                  & Mô tả giảm giá           \\ \hline
        discount\_percent   & decimal               & Phần trăm giảm giá       \\ \hline
        created\_at         & timestamp             & Thời gian tạo            \\ \hline
        deleted\_at         & timestamp             & Thời gian xoá            \\ \hline
        Modified\_at        & timestamp             & Thời gian điều chỉnh     \\ \hline
        \end{tabular}
        \end{table}
\end{itemize}

\section{Phân tích bài toán và đề xuất giải pháp}
\subsection{Phân tích các kiểu kiến trúc hệ thống}
\subsubsection{Kiến trúc monolith}
\noindent \textbf{Định nghĩa} \\[0.3cm]
\noindent Kiến trúc monolith là kiến trúc trong đó tất cả các thành phần của một ứng dụng được đặt trong một đơn vị duy nhất. Đơn vị này thường bị hạn chế trong một phiên bản thời gian chạy duy nhất của ứng dụng. Các ứng dụng truyền thống thường bao gồm giao diện web, lớp dịch vụ và lớp dữ liệu. Trong kiến trúc monolith, các lớp này được kết hợp trên một phiên bản của ứng dụng.\footnote{Website: https://learn.microsoft.com/en-us/training/modules/microservices-architecture/}\\[0.5cm]
\textbf{Lý do sử dụng}\\[0.5cm]
Kiến trúc monolith thường là giải pháp phù hợp cho các ứng dụng nhỏ, nhưng chúng có thể trở nên khó sử dụng khi ứng dụng phát triển. Ban đầu, một ứng dụng nhỏ có thể nhanh chóng trở thành một hệ thống phức tạp, khó mở rộng quy mô, khó triển khai và khó đổi mới.\\[0.5cm]
\textbf{Thách thức}\\[0.5cm]
Tất cả các dịch vụ được chứa trong một đơn vị duy nhất. Sự sắp xếp này mang lại những thách thức khi hoạt động kinh doanh của họ và tải hệ thống tiếp theo phát triển. Một số thách thức này là:
\begin{itemize}
    \item Khó mở rộng quy mô dịch vụ một cách độc lập.
    \item Phức tạp để phát triển và quản lý việc triển khai khi cơ sở mã phát triển, điều này làm chậm quá trình phát hành và triển khai tính năng mới.
    \item Kiến trúc được gắn với một ngăn xếp công nghệ duy nhất, điều này hạn chế sự đổi mới trong các nền tảng và SDK mới.
    \item Cập nhật lược đồ dữ liệu có thể ngày càng khó khăn.
\end{itemize}
Những thách thức này có thể được giải quyết bằng cách xem xét các kiến trúc thay thế, chẳng hạn như kiến trúc microservices.
\subsubsection{Kiến trúc microservices}
\noindent \textbf{Định nghĩa}\\[0.5cm]
Kiến trúc microservice bao gồm các dịch vụ nhỏ, độc lập và được liên kết lỏng lẻo. Mỗi dịch vụ có thể được triển khai và mở rộng quy mô một cách độc lập.\\[0.5cm]
Một microservice đủ nhỏ để một nhóm nhỏ các nhà phát triển có thể viết và duy trì nó. Vì các dịch vụ có thể được triển khai độc lập nên một nhóm có thể cập nhật dịch vụ hiện có mà không cần xây dựng lại và triển khai lại toàn bộ ứng dụng.\\[0.5cm]
Mỗi dịch vụ thường chịu trách nhiệm về dữ liệu riêng của mình. Cấu trúc dữ liệu của nó được tách biệt nên việc nâng cấp hoặc thay đổi lược đồ không phụ thuộc vào các dịch vụ khác. Các yêu cầu về dữ liệu thường được xử lý thông qua API và cung cấp mô hình truy cập nhất quán và được xác định rõ ràng. Chi tiết triển khai nội bộ được ẩn khỏi người tiêu dùng dịch vụ.\\[0.5cm]
Vì mỗi dịch vụ đều độc lập nên chúng có thể sử dụng các nhóm công nghệ, khung và SDK khác nhau. Người ta thường thấy các dịch vụ dựa vào các lệnh gọi REST để liên lạc giữa các dịch vụ bằng cách sử dụng các API được xác định rõ ràng thay vì các lệnh gọi thủ tục từ xa (RPC) hoặc các phương thức liên lạc tùy chỉnh khác.\\[0.5cm]
Kiến trúc vi dịch vụ không phụ thuộc vào công nghệ, nhưng bạn thường thấy các bộ chứa hoặc công nghệ serverless được sử dụng để triển khai chúng. Triển khai liên tục và tích hợp liên tục (CI/CD) thường được sử dụng để tăng tốc độ và chất lượng của các hoạt động phát triển.\\[0.5cm]
\textbf{Lý do sử dụng}\\[0.5cm]
Có một số lợi ích chính đối với kiến trúc microservices:
\begin{itemize}
    \item Nhanh nhẹn
    \item Mã nhỏ, nhóm nhỏ
    \item Sự kết hợp của công nghệ
    \item khả năng phục hồi
    \item Khả năng mở rộng (Scalability)
    \item Cách ly dữ liệu
\end{itemize}
\textbf{Những thách thức}\\[0.5cm]
Có rất nhiều lợi ích đối với kiến trúc microservices, nhưng đó không phải là tất cả. Kiến trúc microservice có những thách thức riêng:
\begin{itemize}
    \item Độ phức tạp
    \item Phát triển và thử nghiệm
    \item Thiếu quản trị
    \item Tắc nghẽn mạng và độ trễ
    \item Toàn vẹn dữ liệu
    \item Sự quản lý
    \item Phiên bản
    \item Bộ kỹ năng    
\end{itemize}
\textbf{Khi nào nên chọn kiến trúc microservices?}\\[0.5cm]
Dựa vào những thông tin trên, kiến trúc microservices sẽ phù hợp trong những tình huống sau
\begin{itemize}
    \item Các ứng dụng lớn đòi hỏi tốc độ phát hành cao.
    \item Các ứng dụng phức tạp cần có khả năng mở rộng cao.
    \item Ứng dụng có miền phong phú hoặc nhiều miền phụ.
    \item Một tổ chức bao gồm các nhóm phát triển nhỏ.    
\end{itemize}
\subsection{Đề xuất giải pháp}
\subsubsection{Kết luận}
Từ việc phân tích các use case trên, chúng ta có thể rút ra được một số kết luận như sau:
\noindent Tất cả các use case đều sử dụng nền tảng cloud để xây dựng hệ thống nhằm đáp ứng tính 
scalability (khả năng mở rộng) và availability (tính sẵn sàng). Một số dịch vụ nền tảng cloud phổ biến nhất hiện nay để quản lý kubernetes là EKS (Elastic Kubernetes Service) và AKS (Azure Kubernetes Service), các dịch vụ này có thể tự động hóa nhiều công việc và tích hợp tốt với các dịch vụ khác trong hệ sinh thái của nhà cung cấp đó nhằm giúp đơn giản hóa việc triển khai và quản lý cụm Kubernetes. Tuy nhiên, chúng ta có thể sử dụng Kubernetes để triển khai ứng dụng của mình trên bất kỳ nơi nào có hỗ trợ Kubernetes, mà không bị ràng buộc bởi một nhà cung cấp dịch vụ nào, hay nói cách khác nó độc lập với hạ tầng cụ thể. 

Bên cạnh đó, Kubernetes còn đảm bảo được hệ thống sẽ đáp ứng được tính scalability (khả năng mở rộng) và availability (tính sẵn sàng) thông qua một số dịch vụ và khái niệm của nó như:
\begin{itemize}
    \item ReplicaSets: Dùng để định nghĩa và duy trì một số lượng replicas (bản sao) của các pod chạy ứng dụng. Điều này giúp đảm bảo tính sẵn sàng của ứng dụng và có thể mở rộng hoặc thu hẹp số lượng replicas dựa trên tải công việc.
    \item Horizontal Pod Autoscaler (HPA): Cho phép tự động thay đổi số lượng replicas của một pod dựa trên các điều kiện như CPU và/hoặc bộ nhớ sử dụng.
    \item Cluster Autoscaler: Tự động mở rộng hoặc thu hẹp cụm Kubernetes bằng cách thêm hoặc giảm số lượng node dựa trên tải công việc.
    \item Load Balancer Services: Sử dụng để phân phối các yêu cầu đến các pod trong cụm, giúp đảm bảo tính sẵn sàng và chia tải đều.
    \item Readiness Probes: Cho phép Kubernetes kiểm tra tính sẵn sàng của mỗi pod trước khi chuyển lưu lượng truy cập đến nó.
    \item Liveness Probes: Kiểm tra xem một pod có hoạt động đúng cách hay không và tự động khởi động lại pod nếu cần.
    \item Pod Disruption Budgets: Giới hạn số lượng pods có thể bị tắt đồng thời trong khi thực hiện các cập nhật hoặc bảo trì để đảm bảo tính sẵn sàng.
\end{itemize}
\noindent Đa phần kiến trúc microservice được ưu tiên áp dụng vì đây là một mô hình ít phụ thuộc vào nhau nên chúng có thể được triển khai, quản lý và mở rộng độc lập với nhau, đồng thời cũng giảm lực cho hệ thống vì chúng không cần phải tương tác trực tiếp với nhau mà thông qua message queue (Apache Kafka, RabbitMQ, ActiveMQ, và Microsoft Azure Service Bus).
\subsubsection{Đề xuất giải pháp}
\noindent Từ các phân tích và kết luận ở trên, nhóm quyết định sẽ xây dựng hệ thống dưới dạng microservice và deloy ứng dụng bằng Kubernetes. Với Kubernetes, ứng dụng sẽ đáp ứng được tính mở rộng và tính sẵn sàng cao thông qua một số dịch vụ mà Kubernetes cung cấp: deloyment, HPA, masternode v.v. Các service của hệ thống sẽ giao tiếp với nhau thông qua message queue.
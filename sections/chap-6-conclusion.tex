\chapter{Tổng kết}
\section{Kết quả đạt được}
\noindent Căn cứ vào mục tiêu, nhiệm vụ của đề tài đã đề ra, nhóm đã thực hiện được những điều sau:
\begin{itemize}
    \item Tiến hành phân tích các yêu cầu cần có của hệ thống, kết hợp với việc so sánh, tham khảo, phân tích các bài viết, ví dụ đã có để đưa ra giải pháp phù hợp với điều kiện thực tế.
    \item Thực hiện một phần hệ thống sớm để kiểm tra tính khả thi, tính đúng đắn của giải pháp ở quy mô nhỏ, từ đó có những thay đổi, cập nhật, điều chỉnh cho phù hợp.
    \item Tối ưu lại giải pháp cho phù hợp với điều kiện kinh tế hiện tại của mỗi cá nhân.
    \item Phân tích các yêu cầu chức năng và phi chức năng của hệ thống.
\end{itemize}
\subsection{Đối với việc tìm hiểu và phân tích nghiệp vụ}
\subsection{Đối với cợ sở lý thuyết và công nghệ}
\subsection{Đối với phân tích và thiết kế hệ thống}
\subsection{Đối với quá trình phát triển ứng dụng}
\noindent Trong suốt giai đoạn Đồ án chuyên ngành, nhóm đã làm việc một cách bài bản, có quy trình, kế hoạch cụ thể và chi tiết.
\begin{itemize}
    \item Nhóm đã thực hiện đầy đủ các giai đoạn: Phân tích, tìm hiểu yêu cầu nghiệp vụ, Thiết kế hệ thống, Triển khai ở quy mô nhỏ.
    \item Công cụ tổ chức và quản lý mã nguồn: Nhóm sử dụng Git và Github để làm tăng hiệu quả công việc.
    \item Về luồng công việc: Nhóm có áp dụng scrum-agile vào việc quản lý, phân chia, theo dõi công việc của các thành viên trong nhóm, tuy nhiên cũng có điều chỉnh cho phù hợp với đặc thù của từng thành viên.
\end{itemize}
\section{Đánh giá giải pháp và kết quả đạt được}
\section{Hướng phát triển đề tài trong tương lai}